\section{Discrete Setting} % (fold)
\label{sec:discrete_setting}

    To turn to our setting, we consider a regular pixel grid of size $N \times M$ and set
        \begin{equation}
            \mathcal{G} = \bigg\{ (i, j) : i = 1, ..., N \,\, \textnormal{and} \,\, j = 1, ..., M \bigg\}.
            \label{eq:pixel_grid}
        \end{equation}
    where the indices $(i, j)$ denote the discrete locations in the pixel grid. And we define our image $u \in X: \mathcal{G} \longrightarrow \mathbb{R}$ where $X \in \mathbb{R}^{N \cdot M}$ is a finite dimensional vector space equipped with the standard inner product
        \begin{equation}
            \langle u, v \rangle_{X} = u^{T}v = \sum_{i = 1}^{N} \sum_{j = 1}^{M} u_{i, j} v_{i, j}, \,\,\, u, v \in X,
            \label{eq:inner_product}
        \end{equation}
    and the standard euclidean norm
        $$
            ||u||_{2} = \big( u^{T} u \big)^{\frac{1}{2}} = \langle u, u \rangle^{\frac{1}{2}}, \,\,\, u \in X.
        $$
    Furthermore, let $Y = X \times X$ be the dual space of $X$ equipped with the inner product defined by
        $$
            \langle p, q \rangle_{Y} = p^{T}q = \sum_{i = 1}^{N} \sum_{j = 1}^{M} p^{1}_{i, j} q^{1}_{i, j} + p^{2}_{i, j} q^{2}_{i, j},
        $$
    with $p_{i, j} = \big(p^{1}_{i, j}, p^{2}_{i, j}\big)^{T}, q_{i, j} = \big(q^{1}_{i, j}, q^{2}_{i, j}\big)^{T} \in Y$ and also equipped with the euclidean norm.

    For simplicity we let our normed vector space $X$ be in $\mathbb{R}^{n}$ and swap the superscript $N \cdot M$ to $n$. If the exact space size is of importance, we will make this clear. Further, we define $\mathbb{R}_{\infty} = \mathbb{R} \cup \{\infty\}$ as the space of the extended real values and $\mathbb{R}_{+} = [0, +\infty)$ as the space of all positive real values including zero.

    Because the total variation is associated with the gradient operator, we want to give some discretization for this operator. As we saw in Equation \ref{eq:nabla_equals_minus_divergence}, the gradient $\nabla$ becomes $- \textnormal{div}$ by swapping it from our function $u$ to the testfunction $\varphi$. This important relation is used in the following and for that we further define the discrete divergence operator.

    \begin{definition}[Discrete gradient operator] % (fold)
    \label{def:discrete_gradient_operator}

        We define the discrete gradient of a vector $u \in X$ by $\nabla_{d}\,u = ((\partial_{i}u)_{i, j}, (\partial_{j}u)_{i, j})^{T}$ using forward differences with Neumann boundary conditions, i.e
            \begin{eqnarray}
                (\partial_{i}u)_{i, j} =
                    \begin{dcases*}
                        u_{i+1, j} - u_{i, j} & \textnormal{if $i < N$}\\
                        0 & \textnormal{if $i = N$}
                    \end{dcases*}
                \label{eq:forward_partial_i} \\
                (\partial_{j}u)_{i, j} =
                    \begin{dcases*}
                        u_{i, j+1} - u_{i, j} & \textnormal{if $j < M$}\\
                        0 & \textnormal{if $j = M$}
                    \end{dcases*}
                \label{eq:forward_partial_j}
            \end{eqnarray}

    \end{definition}
    % definition discrete_gradient_operator (end)

    We choose $\textnormal{div}_{d}: Y \longrightarrow X$ to be the discrete divergence operator, which relates to $\nabla_{d}$ with $- \nabla_{d}^{\ast} = \textnormal{div}_{d}$. That is, for every $p \in Y$ and $u \in X$,
        $$
            \langle \nabla_{d}\, u, p \rangle_{Y} = \langle u, \nabla_{d}^{\ast}\, p \rangle_{X} = - \langle u, \textnormal{div}_{d}\,p \rangle_{X}
        $$
    as in Equation \ref{eq:nabla_equals_minus_divergence}.

    \begin{definition}[Discrete divergence operator] % (fold)
    \label{def:discrete_divergence_operator}

        We define the discrete divergence of a vector $p \in Y$ by $\nabla_{d}^{T} p = \partial_{i}p^{1}_{i, j} + \partial_{j}p^{2}_{i, j}$ using backward differences with Dirichlet boundary conditions, i.e
            \begin{eqnarray}
                (\partial_{i}p^{1})_{i, j} =
                    \begin{dcases*}
                        p^{1}_{i, j} - p^{1}_{i-1, j} & \textnormal{if $1 < i < N$}\\
                        p^{1}_{i, j} & \textnormal{if $i = 1$}\\
                        -p^{1}_{i-1, j} & \textnormal{if $i = N$}
                    \end{dcases*}
                \label{eq:backward_partial_i} \\
                (\partial_{j}p^{2})_{i, j} =
                    \begin{dcases*}
                        p^{2}_{i, j} - p^{2}_{i, j-1} & \textnormal{if $1 < j < M$}\\
                        p^{2}_{i, j} & \textnormal{if $j = 1$}\\
                        -p^{2}_{i, j-1} & \textnormal{if $j = M$}
                    \end{dcases*}
                \label{eq:backward_partial_j}
            \end{eqnarray}

    \end{definition}
    % definition discrete_gradient_operator (end)

    \begin{proposition}[Bound on the norm of $\nabla_{d}$] % (fold)
        \label{prop:bound_on_the_norm}

        The bound on the norm of the proposed discrete linear operators is given by
            \begin{equation}
                L^{2} = ||\nabla_{d}|| = ||\textnormal{div}|| \le 8.
            \end{equation}

    \end{proposition}
    % proposition bound_on_the_norm (end)

    \begin{proof} % (fold)

        With the definitions of \ref{def:discrete_gradient_operator}, \ref{def:discrete_divergence_operator} and \ref{eq:operator_norm} we get
            \begin{eqnarray}
                L &=& ||\nabla_{d}|| = \max_{||x||_{X} \le 1} ||\nabla_{d}\,x||_{Y} \notag \\
                &=& ||\textnormal{div}_{d}|| = \max_{||p||_{Y} \le 1} ||\textnormal{div}_{d}\,p||_{X} \notag \\
                &=& \max_{||p||_{Y} \le 1} \sum_{i = 1}^{N} \sum_{j = 1}^{M} \big( p^{1}_{i, j} - p^{1}_{i-1, j} + p^{2}_{i, j} - p^{2}_{i, j-1} \big)^{2} \notag \\
                &\le& 4 \max_{||p||_{Y} \le 1} \sum_{i = 1}^{N} \sum_{j = 1}^{M} (p^{1}_{i, j})^{2} + (p^{1}_{i-1, j})^{2} + (p^{2}_{i, j})^{2} + (p^{2}_{i, j-1})^{2} \notag \\
                &\le& 8 \max_{||p||_{Y} \le 1} ||p||^{2}_{Y} = 8.
                \label{eq:bound_on_discrete_gradient}
            \end{eqnarray}\qed
    \end{proof}
    % proof (end)

% section discrete_setting (end)