\begin{figure}[ht]
    \centering
    \begin{subfigure}[b]{0.45\textwidth}
        \includegraphics[width=\textwidth]{img/cartooning/realtime/104hepburn500.png}
        \caption{$\alpha = 500, \lambda = 0.4, \gamma = 1$}
    \end{subfigure}
    \begin{subfigure}[b]{0.45\textwidth}
        \includegraphics[width=\textwidth]{img/cartooning/realtime/1004hepburn500.png}
        \caption{$\alpha = 500, \lambda = 0.4, \gamma = 10$}
    \end{subfigure}
    \begin{subfigure}[b]{0.45\textwidth}
        \includegraphics[width=\textwidth]{img/cartooning/realtime/2004hepburn500.png}
        \caption{$\alpha = 500, \lambda = 0.4, \gamma = 20$}
    \end{subfigure}
    \begin{subfigure}[b]{0.45\textwidth}
        \includegraphics[width=\textwidth]{img/cartooning/realtime/5005hepburn500.png}
        \caption{$\alpha = 500, \lambda = 0.5, \gamma = 50$}
    \end{subfigure}
    \caption{Cartooning of Audrey Hepburn with the Mumford-Shah Model. The higher $\lambda$ the closer the outcome of the algrithm to the original image.}
\label{fig:cartooning_hepburn_realtime}
\end{figure}