\documentclass{scrreprt}
\usepackage{amssymb}
\usepackage{mathtools}
\usepackage[utf8]{inputenc}

\begin{document}

\chapter{UR-Model} % (fold)
\label{cha:ur_model}

	\section{UR-Model} % (fold)
	\label{sec:ur_model}

		Die hier angegebenen Formulierungen und Funktionen sind (noch) ohne Herleitung in LaTex gesetzt, jedoch schon schriftlich fest gehalten.

		\subsection{Primales Problem} % (fold)
		\label{sub:primales_problem}
		
			\begin{equation}
				\min\limits_{u, v} \beta ||u - g||_{1} + ||\nabla u||_{1} + \frac{\alpha}{2} ||v - u||_{2}^{2} + ||\nabla v||_{1}.
			\end{equation}

		Da in $||v - u||_{2}^{2}$ beide zu minimierenden Variablen vorkommen, muss man diesen Term in den Funktionen die von $u$ abhängen und auch in den Funktionen die von $v$ abhängen mit einbeziehen. Man erhählt als Funktionen:

			\begin{eqnarray}
				F(u) &=& ||\nabla u||_{1} \\
				G(u) &=& \beta ||u - g||_{1} + \frac{\alpha}{2} ||v - u||_{2}^{2} \\
				F(v) &=& ||\nabla v||_{1} \\
				G(v) &=& \frac{\alpha}{2} ||v - u||_{2}^{2}
			\end{eqnarray}

		Die Idee ist, dass man dieses Problem alternierend löst. Minimiert man nach $u$ ist das $v$ fest, bzw. minimiert man nach $v$ ist das $u$ fest. Also man entrauscht alternierend mit Regularisierern, die gut für Gauß-Rauschen bzw. Salt \& Pepper-Rauschen sind. Man erhält dann
			\begin{eqnarray}
				&\min\limits_{u}& F(u) + G(u) \\
				&\min\limits_{v}& F(v) + G(v)
			\end{eqnarray}

		% subsection primales_problem (end)

		\subsection{Primal-Dual Problem} % (fold)
		\label{sub:primal_dual_problem}

			\begin{eqnarray}
				&\min\limits_{v} \max\limits_{p}& \langle \nabla v, p \rangle - \delta_{P}(p) + \frac{\alpha}{2} ||v - u||_{2}^{2} \\
				&\min\limits_{u} \max\limits_{q}& \langle \nabla u, q \rangle - \delta_{Q}(q) + \beta ||u - g||_{1} + \frac{\alpha}{2} ||v - u||_{2}^{2}
			\end{eqnarray}

		Mit
			$$
				P = \big\{ p \in \mathbb{R}^{n} : ||p||_{\infty} \le 1 \big\}
			$$
		und
			$$
				Q = \big\{ q \in \mathbb{R}^{n} : ||q||_{\infty} \le 1 \big\},
			$$
		wie im ROF-Model für die Legendre-Fenchel konjugierte der totalen Variation.
			
		% subsection primal_dual_problem (end)

		\subsection{Proximity Operatoren} % (fold)
		\label{sub:proximity_operatoren}
			Für die Minimierung über die Variable $v$ erhalte Operatoren aus dem ROF Model:
				\begin{eqnarray}
	                (\textnormal{Id} + \sigma\,\partial\,F^{\ast})^{-1}(\tilde{p}) = P_{l_{2}}(\tilde{p}) = p \Longleftrightarrow p_{i,j} \frac{\tilde{p}_{i, j}}{\max(1, |\tilde{p}_{i, j}|)},\\ \label{eq:proximity_operator_f_star_v}
	                (\textnormal{Id} + \tau\,\partial\,G)^{-1}(\tilde{v}) = v \Longleftrightarrow v_{i,j} = \frac{\tilde{v}_{i,j} + \tau\lambda g}{1 + \tau\sigma}, \label{eq:proximity_operator_g_v}
	            \end{eqnarray}
	        punktweise für alle $i, j$.

		Für den Fall der Variablen $u$ haben wir
		        \begin{eqnarray}
	                q &=& (\textnormal{Id} + \sigma\,\partial\,F^{\ast})^{-1}(\tilde{q}) = P_{l_{2}}(\tilde{q}) \Longleftrightarrow p_{i,j} \frac{\tilde{q}_{i, j}}{\max(1, |\tilde{q}_{i, j}|)},\notag \\
	                u &=& (\textnormal{Id} + \tau\,\partial\,G)^{-1}(\tilde{u}) \Longleftrightarrow u_{i, j} = 
	                    \begin{dcases*}
	                        \frac{\tilde{u} - \tau \beta - \tau \alpha v_{i,j}}{1 - \tau \alpha} & \textnormal{if\, $\tilde{u}_{i,j} - (1 - \tau\alpha)g_{i,j} > \tau\alpha v_{i,j} + \tau\beta$,} \\
	                        \frac{\tilde{u} + \tau \beta - \tau \alpha v_{i,j}}{1 - \tau \alpha} & \textnormal{if\, $\tilde{u}_{i,j} - (1 - \tau\alpha)g_{i,j} < \tau\alpha v_{i,j} - \tau\beta$,} \\
	                        g_{i, j} & \textnormal{if\, $|\tilde{u}_{i,j} - (1 - \tau\alpha)g_{i,j}| \le \tau (\beta + \alpha v_{i, j})$}.
	                    \end{dcases*}\notag
	            \end{eqnarray}
		% subsection proximity_operatoren (end)

	% section ur_model (end)

% chapter ur_model (end)

\end{document}