\section{Total Variation} % (fold)
\label{sec:total_variation}
    
    \begin{definition}[Total Variation] % (fold)
    \label{def:total_variation}

        Let $\Omega$ be an open subset of $\mathbb{R}^{n}$. For a function $u \in L^{1}(\Omega)$, the \textnormal{total variation} of $u$ in $\Omega$ is defined as
            \begin{equation}
                \textnormal{TV}(u) = \sup \bigg\{ -\int_{\Omega} u\, \textnormal{div}\, \varphi\, dx : \varphi \in C^{1}_{0}(\Omega, \mathbb{R}^{n}), |\varphi(x)| \le 1, \forall x \in \Omega \bigg\}.
                \label{eq:total_variation}
            \end{equation}

    \end{definition}
    % definition total_variation (end)

    \begin{example} % (fold)
    \label{prop:u_is_smooth}

        Let $u \in W_{1}^{1}(\Omega)$, then integration by parts and the fact that $\varphi$ having compact supoort gives
            \begin{eqnarray}
                -\langle u, \textnormal{div}\,\varphi \rangle &=& - \int_{\Omega} u\, \textnormal{div}\, \varphi\, dx \notag \\
                &=& - \bigg( \underbrace{\int_{\partial\Omega} u\, \varphi\, d\mathcal{H}^{n-1}}_{= 0} - \int_{\Omega} \nabla\,u\, \varphi\, dx \bigg) = \int_{\Omega} \nabla\,u\,\varphi\,dx \notag \\
                &=& \langle \nabla u, \varphi \rangle,
                \label{eq:nabla_equals_minus_divergence}
            \end{eqnarray}
        for every $\varphi \in C^{1}_{0}(\Omega, \mathbb{R}^{n})$ so that
            \begin{equation}
                \textnormal{TV}(u) = \int_{\Omega} |\nabla\,f| dx.
                \label{eq:tvl1}
            \end{equation}
        The idea to derive Equation \ref{eq:tvl1} is that we first evaluate the case where\\
        $\textnormal{TV}(u) \le \int_{\Omega} |\nabla\,f| dx$. This can easily be seen by using Cauchy-Schwarz-inequality and
            \begin{eqnarray}
                \int_{\Omega} \nabla u \, \varphi \, dx &\le& \bigg| \int_{\Omega} \nabla u \, \varphi dx \bigg| \, dx \notag \\
                &\le& \int_{\Omega} |\nabla u| \, \underbrace{|\varphi|}_{\le 1} \, dx \le \int_{\Omega} |\nabla u| \, dx.
            \end{eqnarray}
        For the inequality in the other direction the key idea would be to set $\varphi = \frac{\nabla u}{|\nabla u|}$. Then clearly, $\varphi = 1$. Since $\varphi \notin C_{0}^{1}$, we would approximate $\varphi$ in $L^{1}$. This space lies close in the space of smooth functions with compact support and for that satisfies the assumptions.

    \end{example}
    % example u_is_smooth (end)

    \begin{remark}
        Note that $W_{1}^{1}(\Omega)$ denotes the $L^{1}$-Sobolev-Space of $\Omega$, which means that $u \in L^{1}(\Omega)$ and $Du \in L^{1}(\Omega)$. Here $Du$ is meant in the sense of distributional derivatives. From now on, we assume that all images $u$ are in $W_{1}^{1}(\Omega)$, where $\Omega$ is the image domain, so that Equation \ref{eq:tvl1} holds. \\
        Further, the total variation is convex and lower-semicontinous. See \cite{Chambolle-et-al-10} for details.
    \end{remark}

    To get a better intuition about the Total Variation, we also want to give the defintion of the variation of functions $u \in [a, b] \subset \mathbb{R}$ with $a, b \in \mathbb{R}$ and $a < b$.

    \begin{definition}[Variation of a function] % (fold)
    \label{def:variation_of_a_function}

        Let $u: \mathbb{R} \longrightarrow \mathbb{R}$ and $a < b$ be real numbers. Then define the variation of $u$ on $[a, b]$ as
            $$
                V^{b}_{a}(u) = \sup \bigg\{ \sum_{i = 1}^{n} |u(x_{i}) - u(x_{i - 1})| : m \in \mathbb{N} \,\textnormal{and}\, a = x_{0} < x_{1} < ... < x_{m} = b \bigg\}.
            $$

    \end{definition}
    % definition variation_of_a_function (end)

    This definition is well known and suited for functions form $\mathbb{R}$ to $\mathbb{R}$. In the sense of measures and integrals this definition is ill-posed [giusti]. But to make variation a bit more vivid this definition is perfect.

    \begin{example} % (fold)
    \label{ex:total_variation_one_d}

        \begin{enumerate}
            \item Let $a, b \in \mathbb{R}$ with $a < b$. If $u: [a, b] \longrightarrow \mathbb{R}$ is monotonically increasing, then for any $a = x_{0} < x_{1} < ... < x_{n} = b$ we observe
                \begin{eqnarray}
                    &&\sum_{i = 1}^{n} |u(x_{i}) - u(x_{i-1})| = \sum_{i = 1}^{n} u(x_{i}) - u(x_{i-1}) \notag \\
                    &=& u(x_{1}) - u(x_{0}) + u(x_{2}) - u(x_{1}) + ... + u(x_{n-1}) - u(x_{n-2}) + u(x_{n}) - u(x_{n-1}) \notag \\
                    &=& u(x_{n}) - u(x_{0}) = u(b) - u(a). \notag
                \end{eqnarray}
            It follows that $V^{b}_{a}(u) = \sup \{u(b) - u(a)\} = u(b) - u(a)$.
            \item Define the function $u: [0, 1] \longrightarrow \mathbb{R}$ by
                \begin{equation}
                    u(x) =
                    \left\{
                        \begin{array}{l l}
                            0,                      & \quad \text{if $x = 0$}, \\
                            x \cos(\frac{\pi}{x}),  & \quad \text{if $x \ne 0$}.
                        \end{array}
                    \right.
                \end{equation}
            This function is continuous and we have $V^{b}_{a}(u) = \infty$. To see this, consider, for each $m \in \mathbb{N}$, the partition $P_{m} = \{ 0, \frac{1}{2m}, \frac{1}{2m-1}, ..., \frac{1}{3}, \frac{1}{2}, 1 \}$. The values of $u$ at the points of this partition are $u(P_{m}) = \{ 0, -\frac{1}{2m}, \frac{1}{2m-1}, ..., -\frac{1}{3}, \frac{1}{2}, -1 \}$.

                % \begin{figure}[ht]
                %     \centering
                %     \begin{tikzpicture}[scale=1.5]
                %         \begin{axis}
                %             \addplot+[domain=0.01:1, black, tension=0.01] {x * cos(deg(pi/x))};
                %             \addplot+[domain=0:0, red, tension=0.3] {0};
                %             \legend{$\sin(x)$}
                %             \addplot[domain=1:3,blue] {-0.5*x*(x-4)};
                %             \draw[dotted] (axis cs:1,-1) -- (axis cs:1,3);
                %             \addplot[soldot] coordinates{(1,1)};
                %             \addplot[holdot] coordinates{(1,1.5)};
                %         \end{axis}
                %     \end{tikzpicture}
                %     \caption{A l.s.c. function.}
                % \end{figure}

                \begin{center}
                     \includegraphics[width=0.8\textwidth]{img/bv3.pdf}
                \end{center}
            For this partition,
                \begin{eqnarray}
                    \sum_{i = 1}^{n} |u(x_{i}) - u(x_{i-1})| &=& |\frac{1}{2m} - 0| + |-\frac{1}{2m-1} - \frac{1}{2m}| + ... + |\frac{1}{2} + \frac{1}{3}| + |-1 - \frac{1}{2}| \notag \\
                    &=& \frac{1}{2m} + \frac{1}{2m-1} + \frac{1}{2m} + \frac{1}{2m-1} + ... + \frac{1}{2} + \frac{1}{3} + 1 + \frac{1}{2} \notag \\
                    &=& 2*(\frac{1}{2m} + \frac{1}{2m-1} + ... + \frac{1}{2}) + 1. \notag
                \end{eqnarray}
            It is known, that the (harmonic) series $\sum_{k = 2}^{\infty} \frac{1}{k}$ diverges. So given any $m$ the partition $P_{m}$ always ensures
                $$
                    V^{b}_{a}(u) = \sup \{ 2 * \sum_{k = 2}^{\infty} \frac{1}{k} + 1 \} = \infty.
                $$
        \end{enumerate}

    \end{example}
    % example total_variation_one_d (end)

    Another concept which later is used in Chapter \ref{cha:a_first_order_primal_dual_algorithm_for_minimizing_the_mumford_shah_functional}
    are (special) functions of bounded variation. Even though we do not need these properties in this chapter it makes sense to define it just here, since it is strongly related to Total Variation as the following definition (see for instance \cite{Giusti}) ensures.

    \begin{definition}[Functions of bounded variation]
    \label{def:functions_of_bounded_variation}

        A function of $u \in L^{1}(\Omega)$ is said to have bounded variation in $\Omega$ if $\textnormal{TV}(u) < \infty$. We define $BV(\Omega)$ as the space of all functions in $L^{1}(\Omega)$ with bounded variation. This space is equipped with the norm
            \begin{equation}
                ||u||_{BV} = ||u||_{L^{1}} + \int_{\Omega} |Du|,
            \end{equation}
        and for that it is a Banach space.
    
    \end{definition}
    % definition functions_of_bounded_variation (end)

    As mentioned the Total Variation is the basis for these class of functions. Related to example \ref{ex:total_variation_one_d} we find that $u \in BV(\mathbb{R})$ in the first example, since we had $V^{b}_{a}(u) = u(b) - u(a) \le \infty$. Furthermore $u$ in the second example is not a function of bounded variation, because $V^{b}_{a}(u) = \infty$.\\
    Additionally, $BV$-functions are proper and therefore we have if $u \in BV(\Omega)$, then also $u \in \Gamma_{0}(\Omega)$, since the total varition is convex and l.s.c.

    \begin{definition}[Special functions of bounded variation \cite{Pock-et-al-iccv09}]
        $SBV$ denotes the special functions of bounded variation, i.e. functions $u$ of bounded variation for which the derivative $Du$ is the sum of an absolutely continuous part $\nabla u \cdot dx$ and a discontinuous singular part $S_{u}$.
    \end{definition}

    These classes of functions play indeed an important role for the Mumford-Shah Functional. But in this thesis the definition is only given for the sake of completeness.

% section total_variation (end)