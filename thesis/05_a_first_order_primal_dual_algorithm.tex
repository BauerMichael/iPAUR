\section{A First-Order Primal-Dual Algorithm} % (fold)
\label{sec:a_firs_order_primal_dual_algorithm}

    Our goal is to solve the saddle-point problem \ref{eq:the_saddle_point_problem}. Therefore, one finds in a series of paper the celebrated (fast) first-order primal-dual algorithm. According to \cite{Chambolle-et-al-10} the idea of this method of solving saddle-point problems in this manner goes back to Arrow and Hurwicz, see also \cite{Arrow-Hurwicz}. For that reason these primal-dual approaches are sometimes also called Arrow-Hurwicz type methods. The first time this algorithm was stated in this framework was probably in \cite{Appleton-Talbot}. The general idea of the proposed algorithm is to do a gradient descent in $u$, since this is the variable of the minimzation problem. And do, simultaneously, a gradient ascent in $p$, because this is the variable of the maximization problem. Choosing time-steps $\sigma, \tau > 0$ one gets
        \begin{eqnarray}
            p^{n+1} = (\textnormal{Id} + \sigma\,\partial\,F^{\ast})^{-1}(p^{n} + \sigma\,Ku^{n}) \notag \\
            u^{n+1} = (\textnormal{Id} + \tau\,\partial\,G)^{-1}(u^{n} - \tau\,K^{\ast}p^{n+1}). \notag
        \end{eqnarray}
    The scheme was first proposed in a paper of Zhu and Chan in \cite{Zhu-Chan}. Unfortunatelly, there is no proof of convergence for it. This makes the approach poor. But in 2009 Pock, Cremers, Bischof and Chambolle published a paper, that we also consider in Chapter \ref{cha:a_first_order_primal_dual_algorithm_for_minimizing_the_mumford_shah_functional}, whose contribution was a provable extension of the above scheme. The idea here was to add an additional extrapolation step to the algorithm, as seen in line three of the primal-dual algorithm \ref{eq:fast_primal_dual_algorithm}. In \cite{Chambolle10afirst-order} Pock and Chambolle generalized this algorithm. They also proposed some variations of the algorithm itself. Depending on the properties of the corresponding functions $F^{\ast}$ and $G$ one can derive a better convergence rate. We will not provide details and only make use of two of the proposed algorithms of \cite{Pock-et-al-iccv09}. In Chapter \ref{cha:applications_to_imaging} we apply the them to different problems in imaging. Further, we will not provide a proof of convergence for these algorithms. For details we refer to \cite{Chambolle10afirst-order} and \cite{Pock-et-al-iccv09}. The general fast primal-dual algorithm is as follows

    \begin{algorithm}[First-Order Primal-Dual Algorithm]
    \label{alg:fast_primal_dual_algorithm}
        Choose $(u^{0}, p^{0}) \in X \times Y$ and let $\bar{u}^{0} = u^{0}$. Further let $\tau, \sigma > 0$ with $\sigma\tau L^{2} \le 1$ and $\theta \in [0, 1]$. Then, we let for each $n \ge 0$
            \begin{equation}
                \left\{ 
                    \begin{array}{l l}
                        p^{n+1} = (\textnormal{Id} + \sigma\,\partial\,F^{\ast})^{-1}(p^{n} + \sigma\,K\bar{u}^{n}) \\
                        u^{n+1} = (\textnormal{Id} + \tau\,\partial\,G)^{-1}(u^{n} - \tau\,K^{\ast}p^{n+1}) \\
                        \bar{u}^{n+1} = u^{n+1} + \theta (u^{n+1} - u^{n}).
                    \end{array}
                \right.
            \label{eq:fast_primal_dual_algorithm}
            \end{equation}
    \end{algorithm}

    These three lines provide a powerful tool. Computing $p^{n} + \sigma\,K\bar{u}^{n}$, respectively, $u^{n} - \tau\,K^{\ast}p^{n+1}$ is easy, since these two are sums of vectors. The last line of this scheme is again the summation of vectors. In the next section we introduce the notation for $K$ and $K^{\ast}$. We will see that, computing the linear operator applied to $u$ and $p$, is again a simple task. Computing the proximity operators in line one and two of the algorithm needs further work and varies within each model. We will compute the proximity operators for each model explicitly in each section.

    Additionally, let us introduce the primal-dual gap, which is strongly related to the weak and strong duality theorems (found for instance in \cite{Geiger-Kanzow}, which is a part of the convergence theorem \ref{the:primal_dual_convergence}.
    \begin{definition}[Primal-Dual Gap] % (fold)
    \label{def:primal_dual_gap}

        Let $u \in X$, $p \in Y$ be the variables of the optimization problem in Equation \ref{eq:the_saddle_point_problem}. Then we define the primal-dual gap of this problem by
            \begin{equation}
                \mathcal{G}(u, p) = \max_{\tilde{p} \in Y} \langle \tilde{p}, Ku \rangle - F^{\ast}(\tilde{p}) + G(u) - \min_{\tilde{u} \in X} \langle p, K\tilde{u} \rangle - F^{\ast}(p) + G(\tilde{u}),
                % \mathcal{G}(u, p) = F(Ku) + G(u) + G^{\ast}(-K^{\ast}p) - F^{\ast}(p),
                \label{eq:primal_dual_gap}
            \end{equation}
        which has the property that $\mathcal{G}(u, p) \ge 0$ for all $u, p$ and equality only holds if and only if $(u, p)$ is a saddle-point. If $\hat{p}$ is a solution of the maximization problem and $\hat{u}$ a solution of the minimization problem the following inequality holds:
            $$
                \mathcal{G}(u, p) \ge \langle \hat{p}, Ku \rangle - F^{\ast}(\hat{p}) + G(u) - \langle p, K\hat{u} \rangle - F^{\ast}(p) + G(\hat{u}) \ge 0
            $$
    \end{definition}
    % definition primal_dual_gap (end)

    For this particular algorithm one can find a convergence theorem in \cite{Chambolle10afirst-order}, which we provide without a proof.

    \begin{theorem} % (fold)
    \label{the:primal_dual_convergence}
        Let $L = ||K||$ and assume Equation \ref{eq:the_saddle_point_problem} has a saddle-point $(\hat{u}, \hat{p})$. Choose $\theta = 1, \tau, \sigma, L^{2} < 1$, and let $(u^{n}, \bar{u}^{n}, p^{n})$ be defined by \ref{eq:fast_primal_dual_algorithm}. Then
            \begin{enumerate}
                \item For any n,
                    $$
                        \frac{||p^{n} - \hat{p}||^{2}}{2\sigma} + \frac{||u^{n} - \hat{u}||^{2}}{2\tau} \le C \bigg( \frac{||p^{0} - \hat{p}||^{2}}{2\sigma} + \frac{||u^{0} - \hat{u}||^{2}}{2\tau} \bigg),
                    $$
                where the constant $C \le (1 - \tau\sigma L^{2})^{-1}$.
                \item If we let $u^{N} = \bigg( \frac{\sum\limits_{n=1}^{N} u^{n}}{N} \bigg)$ and $p^{N} = \bigg( \frac{\sum\limits_{n=1}^{N} p^{n}}{N} \bigg)$, for any bounded $B_{1} \times B_{2} \subset X \times Y$ the restricted primal-dual gap has the following bound:
                    $$
                        \mathcal{G}_{B_{1} \times B_{2}}(u^{N}, p^{N}) \le \frac{D(B_{1}, B_{2})}{N},
                    $$
                where
                    $$
                        D(B_{1}, B_{2}) = \sup_{(u, p) \in B_{1} \times B_{2}} \frac{||u - u^{0}||^{2}}{2\tau} + \frac{||p - p^{0}||^{2}}{2\sigma}.
                    $$
                Moreover, the weak cluster points of $(u^{N}, p^{N})$ are saddle-points of \ref{eq:the_saddle_point_problem}.
                \item If the dimension of the spaces $X$ and $Y$ is finite, then there exists a saddle-point $(u^{\ast}, p^{\ast})$, such that $u^{n} \longrightarrow u^{\ast}$ and $p^{n} \longrightarrow p^{\ast}$.
            \end{enumerate}
    \end{theorem}

    \begin{remark}
        What this theorem states is, that one needs to choose $\tau, \sigma$ carefully by initializing the algorithm. As long as the inequality $\tau\sigma L^{2} < 1$ holds, convergence is guaranteed. The two parameters $\tau, \sigma$ are also called time-steps. The better the choice for these beforhand, the faster the algorithm converges. Unfortunatelly, the two time-step parameters also depend on the choice of the parameters of the underlying model, c.f. chapter \ref{cha:applications_to_imaging}. Estimating the best time-steps is ongoing research. A first approach was a preconditioning scheme, published in \cite{Pock2011}. Of course, it is not best practice having an algorithm which is dependent on manually choosen parameters, which further depend on other parameters. But, on the other hand, two parameters are highly controllable, with respect to the fact that $\sigma$ can be computed from $\tau$ with $\sigma = \frac{1}{\tau * L^{2}}$. Other methods are almost as fast as the primal-dual algorithm, but depend on a couple of parameters, or they are independent of parameter choices and have a slow convergence rate.
    \end{remark}

    As mentioned, the proposed algorithm goes probably back to Arrow-Hurwicz. To derive their original algorithm one would only need to choose $\theta = 0$. Then the last line of algorithm \ref{alg:fast_primal_dual_algorithm} becomes $\bar{u}^{n+1} = u^{n}$. And for that, one can drop $\bar{u}$ and derive the Arrow-Hurwicz method. We state this for the sake of completeness but will not consider it in our computations.

    The convergence rate for our primal-dual algorithm is $\mathcal{O}(\frac{1}{N})$. In the case that one function, $F^{\ast}$ or $G$, is uniformly convex one gets the following variant of the primal-dual algorithm:

    \begin{algorithm}[$F^{\ast}$ or $G$ are uniformly convex]
    \label{alg:f_star_or_g_uniformly_convex}
        Choose $(u^{0}, p^{0}) \in X \times Y$ and let $\bar{u}^{0} = u^{0}$. Further let $\tau_{0}, \sigma_{0}, \gamma > 0$ with $\sigma\tau L^{2} \le 1$. Then, we let for each $n \ge 0$
            \begin{equation}
                \left\{ 
                    \begin{array}{l l}
                        p^{n+1} = (\textnormal{Id} + \sigma_{n}\,\partial\,F^{\ast})^{-1}(p^{n} + \sigma_{n}\,K\bar{u}^{n}) \\
                        u^{n+1} = (\textnormal{Id} + \tau_{n}\,\partial\,G)^{-1}(u^{n} - \tau_{n}\,K^{\ast}p^{n+1}) \\
                        \theta_{n} = \frac{1}{\sqrt{1 + 2\gamma\tau_{n}}}, \, \tau_{n+1} = \theta_{n}\tau_{n}, \, \sigma_{n+1} = \frac{\sigma_{n}}{\theta_{n}} \\
                        \bar{u}^{n+1} = u^{n+1} + \theta_{n} (u^{n+1} - u^{n}).
                    \end{array}
                \right.
            \label{eq:f_star_or_g_uniformly_convex}
            \end{equation}
    \end{algorithm}

    This leads to a convergence rate of $\mathcal{O}(\frac{1}{N^{2}})$. These two types of convergence rates are called sublinear convergence. Fortunatelly, both primal-dual algorithm versions are highly parallelizable on a GPU using the CUDA framework. In Chapter \ref{cha:applications_to_imaging} we provide in-depth details of this.

    % The computations in each line are straightforward. One has to compute sums of vectors and scaled vectors. Since, we will set $K = \nabla$ and for that its conjugate is $-\textnormal{div}$ or equivalently $- \nabla^{T}$, calculating the operator is also easy. The main work needs to be done to find the proximity operators for $G$ and $F^{\ast}$, respectively. They vary within different models. For that, we need to find these operators for each and every model.

    % \begin{algorithm}[$F^{\ast}$ and $G$ are uniformly convex]
    % \label{alg:f_star_and_g_uniformly_convex}
    %     Choose $(u^{0}, p^{0}) \in X \times Y$ and let $\bar{u}^{0} = u^{0}$. Further let $\mu \le \frac{2\sqrt{\gamma\delta}}{L}, \tau = \frac{\mu}{2\gamma}, \sigma = \frac{\mu}{2\delta}$, $\theta \in [\frac{1}{1 + \mu}, 1]$. Then, we let for each $n \ge 0$
    %         \begin{equation}
    %             \left\{ 
    %                 \begin{array}{l l}
    %                     p^{n+1} = (\textnormal{Id} + \sigma\,\partial\,F^{\ast})^{-1}(p^{n} + \sigma\,K\bar{u}^{n}) \\
    %                     u^{n+1} = (\textnormal{Id} + \tau\,\partial\,G)^{-1}(u^{n} - \tau\,K^{\ast}p^{n+1}) \\
    %                     \bar{u}^{n+1} = u^{n+1} + \theta (u^{n+1} - u^{n}).
    %                 \end{array}
    %             \right.
    %         \label{eq:f_star_and_g_uniformly_convex}
    %         \end{equation}
    % \end{algorithm}

% section a_firs_order_primal_dual_algorithm (end)