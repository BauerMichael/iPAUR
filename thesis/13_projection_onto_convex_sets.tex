\section{Projection onto the convex sets and Dykstra's projection algorithm}

    In this section we want to discuss the projections onto the convex sets $C$ and $K$. To project onto $K$, we additionally need Dykstra's project algorithm which we also discuss. We start with the projection onto the set $C$.

    \subsection{Projection onto $C$}

    The projection onto $C$ can efficiently be computed. By definition of the proximity operator we have
        $$
            u = \arg\min_{u \in C} \frac{||u-\tilde{u}||_{2}^{2}}{2} + \tau \delta_{C}(u) = \arg\min_{u \in C} \frac{||u - \tilde{u}||_{2}^{2}}{2}.
        $$
    Assume that $\tilde{u} \in C$. Then the best choice for $u$ is of course $u = \tilde{u}$ itself, since the energy of the minimization problem is equal to zero in this case, for which it is minimal. On the other hand, if $\tilde{u} \notin C$ the euclidean distance or shortest distance to $C = [0, 1]$ is to clip $\tilde{u}$ onto the bound of $C$. This means if $\tilde{u} < 0$ then the shortest distance from $\tilde{u}$ to $u$ is to set $u = 0$. Reversely, if $\tilde{u} > 1$ then the euclidean distance to $C$ is given by setting $u = 1$. This idea is also illustrated in figure (Bild einfuegen!!!). Overall, the projection onto $C$, which is also called clipping or $l_{\infty}$-Projection is given by:

    \begin{algorithm}[Clipping]
        The projection of a vector $u \in \mathbb{R}^{N \times M \times S}$ on the set
            \begin{equation}
                C = \{ u \in X: u(i,j,k) \in [0,1], \,\, u(i, j, 1) = 1, \,\, u(i, j, M) = 0 \}, \label{eq:limits}
            \end{equation}
        is given pointwise by
            \begin{equation}
                u^{n+1}_{i,j,k} = \min\{1, \max \{ 0, u^{n}_{i, j, k} \} \}
            \end{equation}
        for all $i = 1, ..., N, j = 1, ..., M, k = 1, ..., S$.
    \end{algorithm}

    \begin{remark}
        By projecting onto $C$ we also need to take care about the limits in \ref{eq:limits}. For that we set $u(i, j, 1) = 1$ and $u(i, j, S) = 0$ in each projection. Or similarly, using for instance the programming language C++, we would have $u(i, j, 0) = 1$ and $u(i, j, M-1) = 0$.
    \end{remark}

    \subsection{The projection onto $K$} % (fold)
    \label{sub:the_projection_onto_K}

        Projecting onto $K$ is slightly more difficult than the projection onto $C$ since $K$ takes into account local and non-local constraints. In other words, the set $K$ is an intersection of several (convex) sets. The projection onto the intersection of convex sets can be done by Dykstra's projection algorithm. The idea behind this algorithm is to project onto each set $n$ times alternatingly and store the error, which is made in each step. Before the projection in each step is done, the vector which should be projected is reduced by the error made in the previous step. To fully understand this scheme, we first give the definition which was first proposed by Boyle and Dykstra in \cite{dykstra-et-al-aors14}, where one can also find a proof of convergence. Afterwards, we will provide and discuss the algorithm and additionally show an example.

        Consider $P$ convex sets with $\mathbb{R}^{n} \ni X = X_{1} \cap X_{2} \cap ... \cap X_{P}$. Let $\Pi_{i}$ denote the projection onto the i-th set for $i = 1, ..., P$. And let $u_{c} \in \mathbb{R}^{n}$ be the current estimate with $u_{c} \notin X$, $u_{i}^{k} \in \mathbb{R}^{n}$ for $i = 0, ..., P$ and $v_{i}^{k} \in \mathbb{R}^{n}$ for $i = 1, ..., P$ and $k = 1, 2, ...$, where $k$ denotes the number of iterations. Then the algorithm of (Boyle and) Dykstra finds for a point $u_{c} \in \mathbb{R}^{n}$ the (only) $u^{\ast} \in X$ such that

        $$
            ||u^{\ast} - u_{c}||^{2} \le ||u - u_{c}||^{2} \,\,\, \forall u \in X.
        $$

        \begin{algorithm}\label{alg:dykstra}
            For $k = 1, 2, ...$ set $u^{0}_{P} = u_{c}$ and $v^{0}_{i} = 0$ for all $i = 1, ..., P$. Then iterate until convergence (e.g. $||u_{0}^{k} - u_{P}^{k}||_{2} \le \varepsilon$ with $\varepsilon$ small):

            \begin{eqnarray}
                &u_{0}^{k} = &u_{P}^{k-1}, \notag \\
                &\textnormal{for} \,\, &i = 1, 2, ..., P: \notag \\
                &&u_{i}^{k} = \Pi_{i}(u_{i-1}^{k} - v_{i}^{k-1}), \notag \\
                &&v_{i}^{k} = u_{i}^{k} - (u_{i-1}^{k} - v_{i}^{k-1}). \notag
            \end{eqnarray}
        \end{algorithm}

        As mentioned the proof for the theorem of convergence can be found in \cite{dykstra-et-al-aors14}. The theorem is as follows:
        \begin{theorem}[Convergence]
            The sequence $u_{0}^{k}$ in algorithm \ref{alg:dykstra} converges to the (only) point $u \in X$.
        \end{theorem}

        (BILD!!!)

        Recalling algorithm \ref{alg:dykstra}, we see that in each iteration $k$ we project the current vector $u_{i-1}^{k}$ minus the error we made in the previous step, onto the i-th set. Afterwards, we update the error vector $v_{i}^{k}$. As illustrated in (BILD!!!), we project alternatingly on each set. At the end we convergence to the (only) point in $K$.

        Now, that we know how the projection onto the entire set $K$ can be implemented, we need to discuss how a projection of each of the several sets in $K$ looks like. First let us give a decomposition of $K$ into two several sets.
    % subsection the_projection_onto_K (end)

    \subsection{Decomposition of $K$} % (fold)
    \label{sub:decomposition_of_K}
        
        We will decompose the set $K$ into two sets $K_{p}$ and $K_{nl}$, where the first one resembles the local constraint, or more precisely, a parabola constraint and the second one corresponds to the non-local constraint. Overall, we have $K = K_{p} \cap K_{nl}$ and
    % subsection decomposition_of_K (end)

    \begin{equation}
        % &K_{p}& = \bigg\{ p^{3}(i,j,k) \ge \frac{p^{1}(i,j,k)^{2} + p^{2}(i,j,k)^{2}}{4} - \lambda(\frac{k}{M} - f(i,j))^{2} \bigg\} \,\,\, \forall i, j, k \notag \\
        K_{p} = \bigg\{ p^{t}(i, j, k) \ge \frac{||p^{x}(i, j, k)||^{2}}{4} - \lambda(\frac{k}{M} - f(i,j))^{2} \bigg\} \,\,\, \forall i, j, k \label{eq:parabola}
        % \Longleftrightarrow &K_{p}& = \bigg\{ p^{t}(i, j, k) \ge \frac{||p^{x}(i, j, k)||^{2}}{4} - \lambda(\frac{k}{M} - f(i,j))^{2} \bigg\} \,\,\, \forall i, j, k \label{eq:parabola}
    \end{equation}

    where $p^{t}(i, j, k) = p^{3}(i, j, k)$ and $p^{x}(i, j, k) = (p^{1}(i, j, k), p^{2}(i, j, k))^{T}$ as mentioned before. For the non-local constraint we have

    \begin{equation}
        K_{nl} = \bigg\{ \left| \sum_{k_{1} \le k \le k_{2}} p^{x}(i, j, k) \right| \le \nu \bigg\} \,\,\, \forall i, j, k_{1} \le k \le k_{2} \label{eq:nonlocal}
    \end{equation}

    We will now deduce the projection on these two sets. Let us start with the projection onto the parabola.
    %Since $K_{p}$ is an intersection of $P \cdot N \cdot M$ set and $K_{nl}$ is an intersection of $P \cdot N \cdot (\frac{M(M-1)}{2} + M)$ sets we need an algorithm which can handle the projection of a point onto (convex) sets. First let us introduce the projections onto $K_{p}$ and $K_{nl}$.

    \subsection{Projection onto $K_{p}$}

        % Since the projection onto this set also goes pointwise, we have several possibilities to compute the orthogonal projection onto a parabola.\\
        Since the projection onto the set $K_{p}$ is pointwise we want to drop the indices $(i, j, k)$. Note that we do not necessarily need that a $p^{x}$ is an element of $\mathbb{R}^{2}$. The following derivation holds for a larger class of problems namely having $p^{x} \in \mathbb{R}^{n}$.
        % To understand how a projection on a parabola can be computed we want to reformulate \ref{eq:localconst} into an optimization problem.\\
        % \begin{describe}[]
            Let $\alpha > 0$, $p^{x} \in \mathbb{R}^{n}$, $p^{t} \in \mathbb{R}$ %(in our case we would have
            %$$
             %   p^{t} = p^{3}(i, j, k) + \lambda \bigg( \frac{k}{M} - f(i, j)\bigg)^{2}
            %$$
            %for all $i = 1, ..., P$, $j = 1, ..., N$, $k = 1, ..., M$) and
            $p = (p^{x}, p^{t})^{T} \in \mathbb{R}^{n} \times \mathbb{R}$.
            The projection of a $p_{0}$ with $p_{0}^{t} < \alpha ||p_{0}^{x}||_{2}^{2}$ can be written as

            \begin{eqnarray}
                &\min\limits_{p \in \mathbb{R}^{n} \times \mathbb{R}}& \frac{1}{2} ||p - p_{0}||_{2}^{2} \notag \\
                &\textnormal{subject to}& p^{t} \ge \alpha||p^{x}||_{2}^{2} \notag
            \end{eqnarray}

            An equivalent notation for this problem is

            \begin{eqnarray}
                &\min\limits_{p \in \mathbb{R}^{n} \times \mathbb{R}}& f(p) = \frac{(p - p_{0})^{2}}{2} \notag \\
                &\textnormal{subject to}& g(p) = p^{t} - \alpha ||p^{x}||_{2}^{2} \ge 0 \notag
            \end{eqnarray}
        % \end{describe}

        To find the solution of this optimization problem we introduce some auxiliary variable (or Lagrange Multiplier) $\mu \in \mathbb{R}$ and define the Lagrangian as

        \begin{equation}
            \mathcal{L}(x, p, \mu) = f(p) - \mu g(p) = \frac{(p - p_{0})^{2}}{2} - \mu \bigg( p^{t} - \alpha||p^{x}||_{2}^{2} \bigg).
        \end{equation}

        Seeking for the local minima where $\nabla \mathcal{L}(x, p, \mu) = 0$ we even get a global minimum, since our function to optimize is convex, the inequality constraint is convex and the feasible set $\mathbb{R}^{n} \times \mathbb{R}$ is also convex. We set

        \begin{equation}
            \nabla \mathcal{L}(p, \mu) =
            \begin{pmatrix}
                \partial_{p^{x}} \mathcal{L}(p, \mu) \\
                \partial_{p^{t}} \mathcal{L}(p, \mu) \\
                \partial_{\mu} \mathcal{L}(p, \mu)
            \end{pmatrix} = 
            \begin{pmatrix}
                p^{x} - p_{0}^{x} - \mu 2 \alpha p^{x} \\
                p^{t} - p_{0}^{t} - \mu \\
                p^{t} - \alpha||p^{x}||_{2}^{2}
            \end{pmatrix}
            = 0. \label{eq:linearSystem}
        \end{equation}

        That means, we need to solve a linear system. The first equation gives us

        \begin{equation}
            p_{0}^{x} = (\mu 2 \alpha + 1) p^{x} \Longleftrightarrow p^{x} = \frac{p_{0}^{x}}{\mu 2 \alpha + 1}, \label{eq:1stequ}
        \end{equation}

        and the second equation leads us to

        \begin{equation}
            p^{t} = p_{0}^{t} + \mu. \label{eq:2ndequ}
        \end{equation}

        At this point we can solve this system in two ways - if $p_{0}^{t} \ge \alpha ||p_{0}^{x}||_{2}^{2}$ is not true already.
        \begin{enumerate}
            \item By plugging the two equalities \ref{eq:1stequ} and \ref{eq:2ndequ} into the third line of equation \ref{eq:linearSystem}. We get

        \begin{eqnarray}
            p_{0}^{t} + \mu - \alpha \bigg|\bigg|\frac{p_{0}^{x}}{\mu 2 \alpha + 1}\bigg|\bigg|_{2}^{2} = 0 &\Longleftrightarrow& p_{0}^{t} + \mu - \frac{\alpha}{(\mu 2 \alpha + 1)^{2}} ||p_{0}^{x}||_{2}^{2} = 0 \notag \\
            &\Longleftrightarrow& (\mu 2 \alpha + 1)^{2} p_{0}^{t} + (\mu 2 \alpha + 1)^{2} \mu - \alpha ||p_{0}^{x}||_{2}^{2} = 0 \notag \\
            &\Longleftrightarrow& (4 \mu^{2} \alpha^{2} + 4 \mu \alpha + 1) p_{0}^{t} + 4 \mu^{3} \alpha^{2} + 4 \mu^{2} \alpha + \mu - \alpha ||p_{0}^{x}||_{2}^{2} = 0 \notag \\
            &\Longleftrightarrow& 4 \alpha^{2} \mu^{3} + \mu^{2} (4 \alpha^{2} p_{0}^{t} + 4 \alpha) + \mu (4 \alpha p_{0}^{t} + 1) + p_{0}^{t} - \alpha ||p_{0}^{x}||_{2}^{2} = 0. \notag
        \end{eqnarray}

        Solving this equation is straightforward since computing the zeroes can be done by Newton's algorithm with

        \begin{equation}
            \mu^{k+1} = \mu^{k} - \frac{h(\mu)}{h^{'}(\mu)}, \label{eq:newton}
        \end{equation}

        for $k = 1, 2, ...$.

        If we set
        $$
            h(\mu) = 4 \alpha^{2} \mu^{3} + \mu^{2} (4 \alpha^{2} p_{0}^{t} + 4 \alpha) + \mu(4 \alpha p_{0}^{t} + 1) + p_{0}^{t} - \alpha ||p_{0}^{x}||_{2}^{2},
        $$
        we observe
        $$
            h^{'}(\mu) = 12 \alpha^{2} \mu^{2} + 2 \mu (4 \alpha^{2} p_{0}^{t} + 4 \alpha) + (4 \alpha p_{0}^{t} + 1).
        $$
        At the end we are having the update of a $\mu^{k+1}$ with
        \begin{equation}
            \mu^{k+1} = \mu^{k} - \frac{4 \alpha^{2} \mu^{3} + \mu^{2} (4 \alpha^{2} p_{0}^{t} + 4 \alpha) + \mu(4 \alpha p_{0}^{t} + 1) + p_{0}^{t} - \alpha ||p_{0}^{x}||_{2}^{2}}{12 \alpha^{2} \mu^{2} + 2 \mu (4 \alpha^{2} p_{0}^{t} + 4 \alpha) + (4 \alpha p_{0}^{t} + 1)}.
        \end{equation}
        In \cite{Chambolle-et-al-10} they suggest setting $\mu^{0} = \max \{ 0, - \frac{2 p_{0}^{t}}{3} \}$, where they state that Newton's method then converges within 7-10 iterations to a quite accurate solution.\\
        The projected vector $p$ of our problem is then given by
        \begin{equation}
            p = \bigg( \frac{p_{0}^{x}}{\mu 2 \alpha + 1}, p_{0}^{t} + \mu \bigg). \label{eq:newtonSolution}
        \end{equation}

        \item For the second approach we note that \ref{eq:1stequ} and \ref{eq:2ndequ} hold and the third equation in \ref{eq:linearSystem} can only hold if
        \begin{equation}
            p^{t} = \alpha ||p^{x}||_{2}^{2} \Longleftrightarrow \alpha ||p^{x}||_{2}^{2} = p_{0}^{t} + \mu. \label{eq:tmp1}
        \end{equation}
        With \ref{eq:1stequ} we can also compute the solution of $\mu$ by using that

        \begin{eqnarray}
            ||p^{x}||_{2} = \bigg|\bigg| \frac{p_{0}^{x}}{1 + 2 \alpha \mu} \bigg|\bigg|_{2} &\Longleftrightarrow& ||p^{x}||_{2} = \frac{1}{1 + 2 \alpha \mu} ||p_{0}^{x}||_{2} \notag \\
            &\Longleftrightarrow& \frac{1}{1 + 2 \alpha \mu} = \frac{||p^{x}||_{2}}{||p_{0}^{x}||_{2}} \notag \\
            &\Longleftrightarrow& 2 \alpha \mu = \frac{||p_{0}^{x}||_{2}}{||p^{x}||_{2}} - 1 \notag \\
            &\Longleftrightarrow& \mu = \frac{1}{2 \alpha} \bigg( \frac{||p_{0}^{x}||_{2}}{||p^{x}||_{2}} \bigg). \notag
        \end{eqnarray}

        Using the solution of $\mu$ in \ref{eq:tmp1} we get

        \begin{eqnarray}
            \alpha ||p^{x}||_{2}^{2} = p_{0}^{t} + \frac{1}{2 \alpha} \bigg( \frac{||p_{0}^{x}||_{2}}{||p^{x}||_{2}} \bigg) &\overbrace{\Longleftrightarrow}^{\cdot 2 \alpha ||p^{x}||_{2}}& 2 \alpha^{2} ||p^{x}||_{2}^{3} = 2 \alpha ||p^{x}||_{2} p_{0}^{t} + ||p_{0}^{x}||_{2} - 1 \notag \\
            &\Longleftrightarrow& 2 \alpha^{2} ||p^{x}||_{2}^{3} + (1 - 2 \alpha p_{0}^{t}) ||p^{x}||_{2} - ||p_{0}^{x}||_{2} = 0. \notag \\ 
            &\overbrace{\Longleftrightarrow}^{\cdot 4 \alpha}& 8 \alpha^{3} ||p^{x}||_{2}^{3} + 4 \alpha (1 - 2 \alpha p_{0}^{t}) ||p^{x}||_{2} - 4 \alpha ||p_{0}^{x}||_{2} = 0. \notag \\
            &\Longleftrightarrow& (2 \alpha ||p^{x}||_{2})^{3} + 2 (1 - 2 \alpha p_{0}^{t}) 2 \alpha ||p^{x}||_{2} - 4 \alpha ||p_{0}^{x}||_{2} = 0. \notag \\
            &\Longleftrightarrow& t^{3} + 3bt - 2a = 0, \label{eq:cubic}
        \end{eqnarray}

        with $a = 2 \alpha ||p_{0}^{x}||_{2}$, $b = \frac{2}{3}(1 - 2 \alpha p_{0}^{t})$ and $t = 2 \alpha ||p^{x}||_{2}$.\\
        The cubic equation \ref{eq:cubic} in $t$ can efficiently be solved using the analytical formulas for solving cubic equation published by J. P. McKelvey in 1984 in \cite{kelvey-ajp}.

        \end{enumerate}

        The result of the work mentioned is summarized in the following algorithm. Notice that we already computed the factors $a$ and $b$. The others follow with \cite{kelvey-ajp}.

        % After computing the optimal $\mu$ we get the solution to our optimization problem by plugging the Lagrange Multiplier into \ref{eq:1stequ} and \ref{eq:2ndequ} and computing $p^{x}, p^{t}$ respectively.

        % \begin{enumerate}
        %     \item Using Lagrange-Multipliers to rewrite the minimization problem and then applying Newton's method.
        %     \item Following a straightforward formulation proposed in (Evgeny).
        % \end{enumerate}

        % We want to start with the method proposed in (Evgeny).

            \begin{algorithm}
                % Let $\alpha > 0$. For $p_{0}^{x} \in \mathbb{R}^{n}$ and $p_{0} \in \mathbb{R}$ the projection onto the parabola of the form

                % \begin{equation}
                %     \arg \min_{x \in \mathbb{R}^{n}, p \in \mathbb{R}, p \ge \alpha ||x||_{2}^{2}} \frac{(x - p_{0}^{x})^{2}}{2} + \frac{(p - p_{0})^{2}}{2}.
                % \end{equation}

                If already $p_{0}^{t} \ge \alpha ||p_{0}^{x}||_{2}^{2}$, the solution is $(p^{x}, p^{t}) = (p_{0}^{x}, p_{0}^{t})$. Otherwise, with $a = 2 \alpha ||p_{0}^{x}||_{2}$, $b = \frac{2}{3} (1 - 2 \alpha p_{0}^{t})$, and

                    \[
                        d =
                            \begin{dcases*}
                                a^{2} + b^{3} & \textnormal{if $b \ge 0$}\\
                                (a - \sqrt{-b}^{3})(a + \sqrt{-b}^{3}) & \textnormal{else}
                            \end{dcases*}
                    \]

                set

                    \[
                        v =
                            \begin{dcases*}
                                c - \frac{b}{c} \,\, \textnormal{with} \,\, c = \sqrt[3]{a + \sqrt{d}} & \textnormal{if $d \ge 0$}\\
                                2 \sqrt{-b} \cos \bigg( \frac{1}{3} \arccos \frac{a}{\sqrt{-b}^{3}} \bigg) & \textnormal{else}.
                            \end{dcases*}
                    \]

                If $c = 0$ in the first case, set $v = 0$. The solution is then given by

                    \[
                        p^{x} =
                            \begin{dcases*}
                                \frac{v}{2\alpha} \frac{p_{0}^{x}}{||p_{0}^{x}||_{2}} & \textnormal{if $p_{0}^{x} \ne 0$}\\
                                0 & \textnormal{else}
                            \end{dcases*}
                    \]

                and $p^{t} = \alpha ||p^{x}||_{2}^{2}$.
            \end{algorithm}

            % \begin{proof}
            %     Later...
            % \end{proof}

        The above method states that the projection onto the parabola can be done by one cycle of straightforward computations. The implementation of it is quite simple and computation time is fast, which we discuss in detail in chapter \ref{cha:applications_to_imaging}.% But as mentioned there is another approach for this projection.\\

    \subsection{Projection onto $K_{nl}$}

        This set is a combination of non-local constraints, meaning that in a fixed point $(i, j, k)$ you sum up for all $k_{1} \le k \le k_{2}$. For that reason you can not project pointwise. First of all we want to present the algorithm:

        % Let $p_{k} = (p^{1}(i, j, k), p^{2}(i, j, k))^{T}$. Then the projection $p = (p_{1}, ..., p_{M})^{T}$ of a $p_{0} = (p_{0}_{1}, ..., p_{0}_{M})^{T}$ is - for each combination $k_{1} \le k \le k_{2}$ - of the following problem:
        % \begin{algbox}
            \begin{algorithm}[Soft Shrinkage Scheme]\label{alg:softshrinkage}
                Let $p^{i}_{k} = (p^{1}(i, j, k), p^{2}(i, j, k))^{T} \in \mathbb{R}^{2}$, $p^{i} = (p^{i}_{1}, ..., p^{i}_{M})^{T} \in \mathbb{R}^{2 \times M}$ for all $i = 1, 2, ...$. Then the projection $p^{n+1}$ of a $p^{n}$ - for an arbitrary, fixed pair $(k_{1}, k_{2})$ with $1 \le k_{1} \le k \le k_{2} \le M$ - is computed by:

                    \[
                        p^{n+1} =
                            \begin{dcases*}
                                p^{n} + \frac{s - \tilde{s}}{k_{2} - k_{1} + 1} & \textnormal{if $k_{1} \le k \le k_{2}$}, \\
                                p^{n} & \textnormal{else},
                            \end{dcases*}
                    \]

                where $s \in \mathbb{R}^{2}, \tilde{s} \in \mathbb{R}^{2}$ with

                    $$\tilde{s} = \sum_{k_{1} \le k \le k_{2}} p_{k}$$

                and

                    \[
                        s =
                            \begin{dcases*}
                                \tilde{s} & \textnormal{if $||\tilde{s}||_{2} \le \nu$},\\
                                \Pi_{||\cdot||_{2} \le \nu}(\tilde{s}) = \frac{\nu}{||\tilde{s}||_{2}} \tilde{s} & \textnormal{else}.
                            \end{dcases*}
                    \]

                % \begin{equation}
                %     \min\limits_{p_{k} \in \mathbb{R}^{2}} \frac{1}{2} \sum_{k = 1}^{M} ||p_{k} - \tilde{p}_{k}||^{2} \,\,\, \textnormal{s.t.} \,\,\, \bigg{|} \bigg{|} \sum_{k_{1} \le k \le k_{2}} (p^{1}(i, j, k), p^{2}(i, j, k))^{T} \bigg{|} \bigg{|}_{2} \le \nu. \label{eq:minnonlocal}
                % \end{equation}
                % It follows that we have $p_{k} = \tilde{p}_{k}$ for all $k < k_{1}$ and $k > k_{2}$.\\
                % If $k_{1} \le k \le k_{2}$ one can show that we have:
                % \begin{equation}
                %     p_{k} = \tilde{p}_{k} + \frac{s - s_{0}}{k_{2} - k_{1} + 1},
                % \end{equation}
                % where
                % $$s_{0} = \sum_{k_{1} \le k \le k_{2}} \tilde{p}_{k}$$
                % and
                % $$s = \Pi_{||\cdot||_{l^{2}} \le \nu} (s_{0}).$$
            \end{algorithm}
        % \end{algbox}

        Since this procedure needs a clarification, we want to introduce the KKT conditions.

        % \begin{defbox}
            \begin{theorem}[Karush-Kuhn-Tucker Optimality Conditions]
                Consider the optimization problem
                \begin{eqnarray}
                    \min_{x \in \mathbb{R}^{n}} &&f(x) \notag \\
                    \textnormal{s.t.} \,\, && g_{i}(x) \le 0, i = 1, ..., m \notag
                \end{eqnarray}

                Defining the Lagrangian function to this optimization problem we have

                \begin{equation}
                    \mathcal{L}(x, \lambda) = f(x) + \lambda, g(x),
                \end{equation}

                with $\lambda \in \mathbb{R}^{m}$ are called the Lagrange-Multipliers.\\

                Then if the objective function $f: \mathbb{R}^{n} \longrightarrow \mathbb{R}$ and the constraint functions $g_{i}: \mathbb{R} \longrightarrow \mathbb{R}$ are continously differentiable at a point $x^{*}$ then it holds:\\
                $$x^{*} \, \textnormal{is a local minimum} \, \Longleftrightarrow \, \textnormal{there exists a unique} \, \lambda^{*} \, \textnormal{such that}$$
                \begin{itemize}
                    \item Stationarity:
                    $$\nabla_{x} \mathcal{L}(x^{*}, \lambda^{*}) = 0$$
                    \item Complementary Slackness:
                    $$\lambda^{*}_{i} g(x^{*}) = 0 \, \, \, \forall i = 1, ..., m$$
                    \item Primal Feasibility:
                    $$g_{i}(x^{*}) \le 0 \, \, \, \forall i = 1, ..., m$$
                    \item Dual Feasibility:
                    $$\lambda^{*}_{i} \ge 0 \, \, \, \forall i = 1, ..., m$$
                \end{itemize}
            \end{theorem}
        % \end{defbox}

    \begin{proof}[Proof of Algorithm \ref{alg:softshrinkage}]
        Let $p_{k}, \tilde{p}_{k} \in \mathbb{R}^{2}$ and $p, \tilde{p} \in \mathbb{R}^{2 \times M}$ have the same form as in \ref{alg:softshrinkage}. Then for a fixed pair $(k_{1}, k_{2})$ we face the following optimization problem: %$K = k_{2} - k_{1} + 1$, $m = \frac{M(M-1)}{2}$, 
        % $p_{k} \in \mathbb{R}^{2}$, $\tilde{p}_{k} \in \mathbb{R}^{2}$, $p = (p_{1}, ..., p_{M})^{T} \in \mathbb{R}^{2 \times M}$ and $\tilde{p} = (\tilde{p}_{1}, ..., \tilde{p}_{M})^{T} \in \mathbb{R}^{2 \times M}$. Then for a fixed pair $(k_{1}, k_{2})$ we have% have that \ref{eq:minnonlocal} is equivalent to

        \begin{eqnarray}
            &\min\limits_{p \in \mathbb{R}^{2 \times M}}& \frac{1}{2} ||p - \tilde{p}||^{2}_{2} \notag \\
            &\textnormal{subject to}& \bigg{|} \bigg{|} \sum_{k_{1} \le k \le k_{2}} p_{k} \bigg{|} \bigg{|}_{2} \le \nu. \notag
        \end{eqnarray}

        This equivalent to

        \begin{eqnarray}
            &\min\limits_{p \in \mathbb{R}^{2 \times M}}& f(p) \notag \\
            &\textnormal{subject to}& g(p) \le 0, \label{eq:inequalityConstraint}
        \end{eqnarray}

        if we set
            $$f(p) = \sum_{k = 1}^{M} \frac{(p_{k} - \tilde{p}_{k})^{2}}{2}$$
        and
            $$g(p) = \frac{1}{2} \bigg{|} \bigg{|} \sum_{k_{1} \le k \le k_{2}} p_{k} \bigg{|} \bigg{|}_{2}^{2} - \frac{1}{2} \nu^{2}.$$

        This problem states that we try to find the closest $p$ to $\tilde{p}$ whose components fullfil the inequality constraint in \ref{eq:inequalityConstraint}.\\

        % \begin{equation}
        %     \min\limits_{p_{k} \in \mathbb{R}^{2}} \frac{1}{2} \sum_{k = 1}^{M} ||p_{k} - \tilde{p}_{k}||^{2} \,\,\, \textnormal{s.t.} \,\,\, \frac{1}{2} \bigg{|} \bigg{|} \sum_{k_{1} \le k \le k_{2}} (p^{1}(i, j, k), p^{2}(i, j, k))^{T} \bigg{|} \bigg{|}_{2}^{2} - \frac{1}{2} \nu^{2} \le 0 \notag
        % \end{equation}

        Let us now introduce a new variable $\lambda \in \mathbb{R}$ which is called Lagrange Multiplier and observe the Lagrange function (or Lagrangian) with

        % \begin{equation}
        %     \min_{p \in \mathbb{R}^{2 \times M}} f(p) \,\,\, \textnormal{s.t.} \,\,\, g_{i}(p) \le \nu \,\,\, \forall i = 1, ..., m \notag
        % \end{equation}

        % has the following form

        \begin{equation}
            \mathcal{L}(p, \lambda) = f(p) + \lambda g(p)% = \frac{1}{2} \sum_{k = 1}^{M} ||p_{k} - \tilde{p}_{k}||^{2} + \sum_{l = 1}^{K} \lambda_{l} g_{l}(p),
        \end{equation}

        Looking at \underline{Stationarity} condition to find a valid $\lambda^{\ast}$ and $p^{\ast}$ leads us to:

        \begin{equation}
            \nabla_{p} \mathcal{L}(p^{\ast}, \lambda^{\ast}) = \nabla_{p} f(p^{\ast}) + \lambda \nabla_{p} g(p^{\ast}) =
            \underbrace{\begin{pmatrix}
                p^{\ast}_{1} - \tilde{p}_{1} \\
                 \\
                \vdots \\
                 \\
                p^{\ast}_{M} - \tilde{p}_{M}
            \end{pmatrix}}_{\in \mathbb{R}^{M}}
            + \lambda^{\ast}
            \underbrace{\begin{pmatrix}
                0 \\
                \vdots \\
                0 \\
                \sum\limits_{k_{1} \le k \le k_{2}} p^{\ast}_{k} \\
                \vdots \\
                \sum\limits_{k_{1} \le k \le k_{2}} p^{\ast}_{k} \\
                0 \\
                \vdots \\
                0
            \end{pmatrix}}_{\in \mathbb{R}^{M}}
            = 0,
        \end{equation}

        where the zeroes in the last vector are obtain for all components where $k < k_{1}$ and $k > k_{2}$. In these lines we already see that we get

            $$p^{\ast}_{k} = \tilde{p}_{k} \,\,\, \textnormal{if} \,\, k < k_{1} \,\, \textnormal{and} \,\, k > k_{2}.$$

        Let us now take a closer look at the i-th line where $\nabla_{p_{k}} g(p) \ne 0$. Then we have

            \begin{equation}
                p^{\ast}_{i} - \tilde{p}_{i} + \lambda^{\ast} \sum_{k_{1} \le k \le k_{2}} p^{\ast}_{k} = 0. \label{eq:ithRow}
            \end{equation}

        We set % state that the optimal $\lambda^{\ast}$ is of the form

        %     \begin{equation}
        %         \lambda^{\ast} = \frac{1}{k_{2} - k_{1} + 1} \bigg( \frac{\nu}{||\tilde{s}||_{2}} - 1 \bigg). \label{eq:lambda}
        %         % \lambda^{\ast} = \frac{\nu}{||\sum\limits_{k_{1} \le k \le k_{2}} p^{\ast}||_{2}} - 1. \label{eq:lambda}
        %     \end{equation}

        % with

            $$\tilde{s} = \sum\limits_{k_{1} \le k \le k_{2}} \tilde{p}_{k}$$

        and define a $s \in \mathbb{R}^{2}$ as

            \[
                s =
                    \begin{dcases*}
                        \tilde{s} & \textnormal{if $||\tilde{s}||_{2} \le \nu$},\\
                        \Pi_{||\cdot||_{2} \le \nu}(\tilde{s}) = \frac{\nu}{||\tilde{s}||_{2}} \tilde{s} & \textnormal{else}.
                    \end{dcases*}
            \]

        \newpage

        For finding the optimal values $p^{\ast}_{k}$ we want to distinguish between two cases:

        \begin{enumerate}
            \item $||\sum\limits_{k_{1} \le k \le k_{2}} \tilde{p}_{k}||_{2} \le \nu$:\\
            Setting $p^{\ast}_{k} = \tilde{p}_{k} + \frac{s - \tilde{s}}{k_{2} - k_{1} + 1} = \tilde{p}_{k}$ for all $k_{1} \le k \le k_{2}$ and $s = \tilde{s}$ leads to
                \begin{equation}
                    \frac{1}{2} ||\sum_{k_{1} \le k \le k_{2}} p^{\ast}_{k}||_{2}^{2} - \frac{1}{2} \nu^{2} \le 0, \label{eq:caseOne}
                \end{equation}
            % which validates the \underline{Primal Feasibility} condition.
            It follows for the \underline{Complementary Slackness} condition that we can choose an arbitrary $\lambda^{\ast} \ge 0$ if we have equality in \ref{eq:caseOne} and $\lambda^{\ast} = 0$ else to derive the \underline{Dual Feasibility} condition. We set
            % The \underline{Complementary Slackness} condition claims that in this case% validates the \underline{Dual Feasibility} condition in this case because having equality in \ref{eq:caseOne} leads to
                $$\lambda^{\ast} = 0.$$

            % since $||\tilde{s}||_{2} \le \nu$.
            % which is equivalent to
                % ||\tilde{s}||_{2} = ||\sum\limits_{k_{1} \le k \le k_{2}} \tilde{p}^{k}||_{2} = 
                % $$\frac{\nu}{||\tilde{s}||_{2}} - 1 = 0 \Longleftrightarrow ||\tilde{s}||_{2} = \nu.$$
            % On the other hand if the inequality holds we have
            %     % ||\tilde{s}||_{2} = ||\sum\limits_{k_{1} \le k \le k_{2}} \tilde{p}^{k}||_{2} = 
            %     $$||\sum\limits_{k_{1} \le k \le k_{2}} p^{\ast}_{k}||_{2} < \nu \Longrightarrow \lambda^{\ast} = \underbrace{\frac{\nu}{||\sum\limits_{k_{1} \le k \le k_{2}} p^{\ast}_{k}||_{2}}}_{> 1} - 1 > 0.$$
            % Then $\lambda^{\ast}$ additionally satisfies the \underline{Complementary Slackness} condition.
            \item $||\sum\limits_{k_{1} \le k \le k_{2}} \tilde{p}_{k}||_{2} > \nu$:\\
            % If the equality holds we know that $s(\tilde{s})$
            Here, we would violate the \underline{Primal Feasibility} condition if we set $p^{\ast}_{k} = \tilde{p}_{k}$ for all $k_{1} \le k \le k_{2}$. For this reason we choose
                $$p^{\ast}_{k} = \tilde{p}_{k} + \frac{s - \tilde{s}}{k_{2} - k_{1} + 1} \,\,\,\,\,\, \forall k_{1} \le k \le k_{2}$$
                % $$p^{\ast}_{k} = \frac{\nu}{||\tilde{s}||_{2}} \tilde{p}_{k} \,\,\,\,\,\, \forall k_{1} \le k \le k_{2}$$
            We observe for the \underline{Primal Feasibility} condition that
                \begin{eqnarray}
                    \frac{1}{2} ||\sum_{k_{1} \le k \le k_{2}} p^{\ast}_{k}||_{2}^{2} - \frac{1}{2} \nu^{2} &=& \label{eq:caseTwo} \\
                    &=& \frac{1}{2(k_{2} - k_{1} + 1)} ||\sum_{k_{1} \le k \le k_{2}} \tilde{p}_{k} + (s - \tilde{s})||_{2}^{2} - \frac{1}{2} \nu^{2}\notag \\
                    &=& \frac{1}{2} ||\sum_{k_{1} \le k \le k_{2}} \tilde{s} + s - \tilde{s}||_{2}^{2} - \frac{1}{2} \nu^{2} = \frac{1}{2} \underbrace{||s||_{2}^{2}}_{= \nu^{2}} - \frac{1}{2} \nu^{2} = 0. \notag
                \end{eqnarray}
            Using this equality we already see that \underline{Complementary Slackness} is fullfild with a \underline{dual feasible} $\lambda^{\ast} \ge 0$. Since we set $p^{\ast}_{k} = \tilde{p}_{k} + \frac{s - \tilde{s}}{k_{2} - k_{1} + 1}$ we can plug $p^{\ast}$ into \ref{eq:ithRow} to derive $\lambda^{\ast}$. It follows %We still need to proof that \ref{eq:lambda} is valid with this choice of $p^{\ast}_{k}$.

                \begin{eqnarray}
                    &&p^{\ast}_{i} - \tilde{p}_{i} + \frac{\lambda^{\ast}}{k_{2} - k_{1} + 1} \sum_{k_{1} \le k \le k_{2}} p^{\ast}_{k} = \notag \\
                    &=& \tilde{p}_{i} + \frac{s - \tilde{s}}{k_{2} - k_{1} + 1} - \tilde{p}_{i} + \frac{\lambda^{\ast}}{k_{2} - k_{1} + 1} \sum_{k_{1} \le k \le k_{2}} \tilde{p}_{k} + \frac{s - \tilde{s}}{k_{2} - k_{1} + 1} \notag \\
                    &=& \frac{s - \tilde{s}}{k_{2} - k_{1} + 1} + \frac{\lambda^{\ast}}{k_{2} - k_{1} + 1} \bigg( \underbrace{\sum_{k_{1} \le k \le k_{2}} \tilde{p}_{k}}_{= \tilde{s}} + \underbrace{\sum_{k_{1} \le k \le k_{2}} \frac{s - \tilde{s}}{k_{2} - k_{1} + 1}}_{= \frac{(k_{2} - k_{1} + 1) (s - \tilde{s})}{k_{2} - k_{1} + 1} = s - \tilde{s}} \bigg) \notag \\
                    &=& \frac{s - \tilde{s}}{k_{2} - k_{1} + 1} + \frac{\lambda^{\ast}}{k_{2} - k_{1} + 1} (\tilde{s} + s - \tilde{s}) \notag \\
                    &=& \frac{s - \tilde{s}}{k_{2} - k_{1} + 1} + \frac{\lambda^{\ast}}{k_{2} - k_{1} + 1}s \notag \\
                    &=& \frac{1}{k_{2} - k_{1} + 1} (s - \tilde{s} + \lambda^{\ast}s) = 0 \label{eq:equalsZero}
                \end{eqnarray}

            Now we can solve for $\lambda^{\ast}$ using that the last equation \ref{eq:equalsZero} is equivalent to

                \begin{eqnarray}
                    (s - \tilde{s} + \lambda^{\ast} s) = 0 &\overbrace{\Longleftrightarrow}^{||\tilde{s}||_{2} > \nu}& \bigg(\frac{\nu}{||\tilde{s}||_{2}}\tilde{s} - \tilde{s} + \lambda^{\ast} \frac{\nu}{||\tilde{s}||_{2}}\tilde{s} \bigg) = 0 \notag \\
                    &\Longleftrightarrow& \tilde{s} \bigg( \frac{\nu}{||\tilde{s}||_{2}} - 1 + \lambda^{\ast} \frac{\nu}{||\tilde{s}||_{2}} \bigg) = 0 \notag \\
                    &\Longleftrightarrow& \frac{\nu}{||\tilde{s}||_{2}} - 1 + \lambda^{\ast} \frac{\nu}{||\tilde{s}||_{2}} = 0 \notag \\
                    &\Longleftrightarrow& \lambda^{\ast} \frac{\nu}{||\tilde{s}||_{2}} = 1 - \frac{\nu}{||\tilde{s}||_{2}} \notag \\
                    &\Longleftrightarrow& \lambda^{\ast} = \underbrace{\frac{||\tilde{s}||_{2}}{\nu}}_{> 1} - 1 > 0. \notag
                \end{eqnarray}

            This satisfies the \underline{Dual Feasibility} condition.

                % $$||\tilde{s}||_{2} = ||\sum\limits_{k_{1} \le k \le k_{2}} \tilde{p}_{k}||_{2} = ||\sum\limits_{k_{1} \le k \le k_{2}} \frac{||\tilde{s}||_{2}}{\nu} p^{\ast}||_{2} = ||\tilde{s}||_{2} =  \nu \Longrightarrow \lambda^{\ast} = 1 - 1 = 0.$$

                % $$\nu = ||\sum\limits_{k_{1} \le k \le k_{2}} p^{\ast}_{k}||_{2} = ||\frac{\nu}{||\tilde{s}||_{2}} \sum\limits_{k_{1} \le k \le k_{2}} \tilde{p}^{k}||_{2} = ||\tilde{s}||_{2} =  \nu \Longrightarrow \lambda^{\ast} = 1 - 1 = 0.$$

                % \begin{equation}
                %     p^{\ast}_{i} - \tilde{p}_{i} + \frac{\lambda^{\ast}}{k_{2} - k_{1} + 1} \tilde{s} = \frac{\nu}{||\tilde{s}||_{2}} \tilde{p}_{i} - \tilde{p}_{i} + \frac{1}{k_{2} - k_{1} + 1} \bigg( \frac{\nu}{||\tilde{s}||_{2}} - 1 \bigg) \tilde{s}
                % \end{equation}
            % First, we define a two-dimensional vector $s$ with
            %     \begin{equation}
            %         s(p) = \Pi_{||\cdot||_{2} \le \nu} (p) = \frac{\nu}{||p||_{2}} p.
            %     \end{equation}
            % If the equality in \ref{eq:caseTwo} holds. Then $||$
        \end{enumerate}

        With the choices of $\tilde{s}, s$ and

            $$p^{\ast}_{k} = \tilde{p}_{k} + \frac{s - \tilde{s}}{k_{2} - k_{1} + 1} \,\,\,\,\,\, \forall k_{1} \le k \le k_{2}$$

        we solved \ref{eq:ithRow} (\underline{Stationarity}). Applying this procedure to each combination of the $(k_{1}, k_{2})$ and replacing $p^{\ast}$ by $p^{n+1}$, $\tilde{p}$ by $p^{n}$ respectively, we observe our algorithm.
        \qed
    \end{proof}

        % with $\lambda \in \mathbb{R}^{m}$ and $\lambda_{i} \ge 0$ for all $i = 1, ..., m$.
        % With this and the fact that the inequality constraint also holds for all $i = 1, ..., m$ the \underline{Primal} and the \underline{Dual Feasibility} are already fullfild.\\ 

        % Applying \underline{Stationarity} condition we have:

        % \begin{equation}
        %     \nabla \mathcal{L}(x^{*}, \lambda^{*}) = 0 \Longleftrightarrow - \nabla f(p^{*}) = \sum_{l = 1}^{m} \lambda^{*}_{l} \nabla g_{l}(p^{*}).
        % \end{equation}

        % And \underline{Complementary Slackness} telling us that

        % \begin{equation}
        %     \lambda^{*} g(p^{*}) = 0.% \bigg( \frac{1}{2} \bigg{|} \bigg{|} \sum_{k_{1} \le k \le k_{2}} (p^{1}(i, j, k), p^{2}(i, j, k))^{T} \bigg{|} \bigg{|}_{2}^{2} - \frac{1}{2} \nu^{2} \bigg) = 0.
        % \end{equation}

% Back to the convex set $K$ we find that this set is the intersection of several sets. In \ref{eq:localconst} you find local constraints, more precisely parabola constraints. This constraint is pixel wise. If the third component of a vector $p$ namely $p^{3}$ satisfies this inequality the vector $p$ already lies in the constraint set \ref{eq:localconst}. If not some computations need to be done. We'll come back to that later. \\
% On the other hand the set in \ref{eq:nonlocalconst} is more complicated and also needs special treatment. This set takes not only values from one fixed voxel into account but sums up several voxel from a range $k_{1}$ to $k_{2}$. In the end we need to have that $p \in K$ and $p$ must satisfy both constraints. \\

