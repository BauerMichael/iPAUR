\section{Primal and Dual Formulation and the Proximity Operators} % (fold)
\label{sub:the_mumford_shah_model_as_saddle_point_problem}

	Recalling equation \ref{eq:mumford_shah_saddle_point_problem} and setting $A = \nabla$, we have that a minimizer of the Mumford-Shah Functional can be computed by solving
		\begin{equation}
			\min_{u \in C} \max_{p \in K} \langle \nabla u, p \rangle.
			\label{eq:primal_dual_mumford_shah}
		\end{equation}

	As in the chapter before we first want to formulate this saddle-point problem in the primal and the dual version. Afterwards, we want to compute the proximity operators for the corresponding functions.

	We can write equation \ref{eq:primal_dual_mumford_shah} in a slightly different way, since we can simply add the indicator functions of the corresponding sets $C$ and $K$. We observe
		\begin{equation}
			\min_{u \in C} \max_{p \in K} \langle \nabla u, p \rangle - \delta_{K}(p) + \delta_{C}(u).
			\label{eq:primal_dual_mumford_shah_complete}
		\end{equation}
	But then, we immediately can determine our functions $F^{\ast}$ and $G$, which are given by
		$$
			F^{\ast}(p) = \delta_{K}(p) \,\,\, \textnormal{and} \,\,\, G(u) = \delta_{C}(u).
		$$
    With this, we first want to take a look at the primal formulation of the Mumford-Shah model. Therefore, it is left to compute $F$ from $F^{\ast}$. Assume the function $F(g)$ to be convex, then the Legendre-Fenchel conjugate is given by
        $$
            F^{\ast}(p) = \sup_{u \in C} \langle g, p \rangle - F(g).
        $$
    For $F$ convex we observe on the other hand
        $$
            F(g) = \sup_{p \in Y} \langle g, p \rangle - F^{\ast}(p) = \sup_{p \in Y} \langle g, p \rangle - \delta_{K}(p).
        $$
    We want to compute $F(g)$ and show that the function $F$ is indeed convex. Computing $F$ means solving
        $$
            \sup_{p \in Y} \langle g, p \rangle - \delta_{K}(p) = \sup_{p \in K} \langle g, p \rangle.
        $$
    % We drop the indicator function, because in the case $p \notin K$ the function would be $\delta_{K}(p) = \infty$, and for that we would not attain the supremum. Also note that we are seeking for the supremum for all $p \in K$, for that the indicator function is zero.
    We see that, since $K$ is finite dimensional, we can change the supremum to the maximum and observe
        $$
            \max_{p \in K} \langle g, p \rangle \Longleftrightarrow 0 \in \partial \left( g, p \right) \Longleftrightarrow 0 \in g \Longleftrightarrow g = 0.
        $$
    But this means nothing that, if $g = 0$ then $F(g) = 0$. If $g < 0$ to attain a maximal value the only choice to let $p$ go to $- \infty$. But if $g > 0$ then $p \longleftarrow \infty$. For an arbitray argument of $F$, we have
        $$
            F(u) =
                \begin{dcases*}
                    0 & \textnormal{if $u = 0$,} \\
                    \infty & \textnormal{else.}
                \end{dcases*}
        $$
    This is nothing but the indicator function of a single point. Then for the primal formulation we get
        \begin{eqnarray}
            \min_{u \in C} F(\nabla u) + G(u) &\Longleftrightarrow& \min_{u \in C} G(u) = 0\\
            &\textnormal{s.t.}& \,\,\, \nabla u = 0.
            \label{eq:primal_mumford_shah}
        \end{eqnarray}

	Finally, we also want to evaluate the dual formulation of the Mumford-Shah saddle-point problem. The dual of the saddle-point problem was given by
		$$
            \max_{p \in Y} -(G^{\ast}(-K^{\ast}p) + F^{\ast}(p)).
        $$
    We need to compute the Legendre-Fenchel conjugate of $G$. Since this is the indicator function of the set $C$ we can make use of Example \ref{ex:legendre_fenchel_conjugate_example} 1. and see that $G^{\ast}(p) = \delta^{\ast}_{C}(p) = \sup\limits_{u \in C} \langle p, u \rangle$, which is the support function of $C$. Overall, we obtain the dual problem by
    	\begin{equation}
    		\max_{p \in Y} -(G^{\ast}(-K^{\ast}p) + F^{\ast}(p)) = \max_{p \in K \subset Y} -(\langle \nabla^{T}p, u \rangle + \delta_{K}(p)) = \max_{p \in K} -\langle \nabla^{T}p, u \rangle.
    	\label{eq:dual_mumford_shah}
    	\end{equation}
    
    It is left to compute the proximity operators. Since, the proximal operator for a indicator function is a euclidean projection onto the corresponding set, we need to describe how a projection on $C$ and $K$ can be derived.
    %For the Mumford-Shah model we need to compute projections onto convex sets. Let us note that $F^{\ast}(p) = \delta_{K}(p)$ and $G(u) = \delta_{C}(u)$. From example \ref{ex:proximity_operator} we know that the proximity operator of the indicator function is a projection onto the corresponding convex set. This implies, that the proximity operators for $F^{\ast}$ and $G$ are projections onto $K$ and $C$, respectively.
    For that we want to rewrite our primal-dual algorithm to be consistent with \cite{Pock-et-al-iccv09}. We observe
    % To solve the primal formulation one only needs to solve a linear equation, namely $\nabla u = 0$, meaning, that the minimum can only be attained if $F(\nabla u) = 0$ and for that $G(u) = 0$, otherwise $G$ would be $\infty$.

    % The question we need to answer is, for what this formulation here is useful for. We earlier introduced the Primal-Dual Gap. And we saw that this gap vanishes if and only if $(u, p)$ are saddle-points. In other words, if we let our iterations $n$ go to $\infty$, then
    % 	$$
    % 		\max_{\tilde{p} \in Y} \langle \tilde{p}, \nabla u \rangle - F^{\ast}(\tilde{p}) + G(u) - \min_{\tilde{u} \in X} \langle p, \nabla \tilde{u} \rangle - F^{\ast}(p) + G(\tilde{u}) = 0.
    % 	$$
    % To compute the Primal-Dual Gap in each iteration step, we need to solve the primal and the dual formulation ,respectively. 

    % We are set up to to compute the proximity operators of the Mumford-Shah model.

% section the_mumford_shah_model_as_saddle_point_problem (end)

% \section{The Proximity Operators for the Mumford-Shah Model} % (fold)
% \label{sub:the_proximity_operators_for_the_mumford_shah_model}
	
	% In this section we will see, that for the Mumford-Shah model we need to compute projections onto convex sets. Let us note that $F^{\ast}(p) = \delta_{K}(p)$ and $G(u) = \delta_{C}(u)$. From example \ref{ex:proximity_operator} we know that the proximity operator of the indicator function is a projection onto the corresponding convex set. This implies, that the proximity operators for $F^{\ast}$ and $G$ are projections onto $K$ and $C$, respectively. For that we want to rewrite our primal-dual algorithm to be consistent with \cite{Pock-et-al-iccv09}. We observe

		\begin{algorithm}\label{alg:primal_dual_cremers}
            Choose $(u^{0}, p^{0}) \in C \times K$ and let $\bar{u}^{0} = u^{0}$. We choose $\tau, \sigma > 0$. Then, we let for each $n \ge 0$
                \begin{equation}
                    \left\{ 
                        \begin{array}{l l}
                          p^{n+1} = \Pi_{K} (p^{n} + \sigma K \bar{u}^{n}) \\
                          u^{n+1} = \Pi_{C} (u^{n} - \tau K^{*} p^{n+1}) \\
                          \bar{u}^{n+1} = 2u^{n+1} - u^{n}.
                        \end{array}
                    \right.
                \end{equation}
        \end{algorithm}

    Here we denote $\Pi_{K}$ and $\Pi_{C}$ as the (euclidean) projections on the sets $K$ and $C$. How we compute such projections, especially onto the set $K$, will be discussed in the next section.
    
% section the_proximity_operators_for_the_mumford_shah_model (end)

% section the_mumford_shah_functional (end)