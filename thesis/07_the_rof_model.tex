Since, there exists a couple of models to approximate an input image $g$ by a function $u$, we want to make some conventions about this beforehand. Our models are based on the idea that a given image $g$ consists of two things: data and noise. This can be expressed by
    $$
        g = g_{d} + g_{n},
    $$
where $g_{d}$ denotes the data of the image and $g_{n}$ the noise which should be removed.

An example model to efficiently remove Gaussian noise from an input image $g$ would be the so called ROF model. Before introducing this model in detail, let us state one general property of all our underlying models. What they all have in common is the idea how they are set up:
    $$
        \textnormal{Model} = \textnormal{Data Fidelity Term} + \textnormal{Regularizer Term}.
    $$
The data fidelity term assures that the approximation $u$ is as close to the input image $g$ as possible. For instance, we set $G(u) = \frac{1}{2} ||u - g||_{2}^{2}$ having the quadratic, euclidean distance as the data fidelity term. As a quadratic norm function this term is convex. For the regularizer we use convex terms, which are easy to handle, but also highly non-convex regularizers, like in the Mumford-Shah Functional, discussed in Chapter \ref{cha:a_first_order_primal_dual_algorithm_for_minimizing_the_mumford_shah_functional}.

\section{The ROF Model} % (fold)
\label{sec:the_rof_model}
    
    The first model we consider in this thesis is the ROF Model, named after Leonid I. Rudin, Stanley Osher and Emad Fatemi. They first proposed this model in 1992 in \cite{ROF}. It is the prototype when talking about variational methods in image processing. Before we define this model in details, we define two important norms. The discrete isotropic total variation norm is given by
        \begin{equation}
            ||\nabla \, u||_{1} = \sum_{i = 1}^{N} \sum_{j = 1}^{M} |(\nabla \, u)_{i, j}|, \,\,\, \textnormal{where} \,\,\, |(\nabla \, u)_{i, j}| = \sqrt{((\nabla \, u)^{1}_{i, j})^{2} + ((\nabla \, u)^{2}_{i, j})^{2}}.
        \label{eq:isotropic_total_variation_norm}
        \end{equation}

    Additionally, we define the discrete maximum norm by
        \begin{equation}
            ||p||_{\infty} = \max_{i, j} |p_{i, j}|, \,\,\, \textnormal{where} \,\,\, |p_{i, j}| = \sqrt{(p^{1}_{i, j})^{2} + (p^{2}_{i, j})^{2}}.
        \end{equation}

    \begin{definition}[ROF Model] % (fold)
    \label{def:the_rof_model}

        Let $\Omega \in \mathbb{R}^{d}$ be the $d$-dimensional image domain, $u \in W_{1}^{1}(\Omega)$ and $g \in L^{1}(\Omega)$ an (noisy) input image. Then the ROF model is defined as the minimization problem
            \begin{equation}
                \min_{u} E_{ROF}(u) = \min_{u} \textnormal{TV}(u) + \frac{\lambda}{2} \int_{\Omega} |u - g|^{2} \, dx = \min_{u} \int_{\Omega} |\nabla \, u| \, dx + \frac{\lambda}{2} \int_{\Omega} |u - g|^{2} \, dx,
                \label{eq:the_rof_model}
            \end{equation}

    \end{definition}
    % definition the_rof_model (end)

    The appearing parameter $\lambda$ is used to handle the tradeoff between the regularizer, namely the total variation, and the data fidelity term. Using a larger value for $\lambda$ one gets an approximation $u$, which is closer to the input image $g$. If we choose $\lambda$ to be small, then the weighting of the regularizer is higher and for that the approximation $u$ is smoother.
    %and sharper edges are favored. Sometimes, this model is also called the TVL2 Model. The name is a compound of the regularizer, the total variation (TV) and the norm for the data fidelity term, the $L_{2}$ norm.

    Reformulation of Equation \ref{eq:the_rof_model} into a discrete setting leads us to:
        \begin{equation}
            \min_{u \in X} E_{ROF}(u) = \min_{u \in X} ||\nabla \, u||_{1} + \frac{\lambda}{2} ||u - g||_{2}^{2},
        \label{eq:discrete_rof_model}
        \end{equation}
    with $u \in X$ being the unknown approximation and $g \in X$ the given noisy data.

    \begin{remark} % (fold)
        In some literature, for instance \cite{Chambolle10afirst-order}, one finds a multiplicative factor $h^{2}$ in Equation \ref{eq:discrete_rof_model}. This factor is due to discretization. Since we assume our image domain $\Omega$ having its pixel values as discrete locations, we do not make use of the additional factor in none of our models. Note, that the factor $h^{2}$ only rescales the energy function $E_{ROF}(u)$ and does not change the solution of the minimization problems.
    \end{remark}
    % remark (end)

    \subsection{ROF Model as Saddle-Point Problem} % (fold)
    \label{sub:rof_model_as_saddle_point_problem}

        Let us now rewrite the minimization problem in Equation \ref{eq:discrete_rof_model} to derive a saddle-point formulation. For that, we set $F(\nabla u) = ||\nabla \, u||_{1}$ and $G(u) = \frac{\lambda}{2} ||u - g||_{2}^{2}$. Hence, we are facing the following optimization problem
            \begin{equation}
                \min_{u \in X}\,\, F(\nabla u) + G(u) = \min_{u \in X}\,\, ||\nabla \, u||_{1} + \frac{\lambda}{2} ||u - g||_{2}^{2}.
            \label{eq:primal_rof_problem}
            \end{equation}
        Applying the Legendre-Fenchel conjugate, one has $F^{\ast\ast}(\nabla u) = F(\nabla u) = \sup\limits_{p \in Y} \langle p, \nabla \, u \rangle_{Y} - F^{\ast}(p)$, and we observe the saddle-point problem
            \begin{equation}
                \min_{u \in X}\, \max_{p \in Y}\,\, \langle p, \nabla \, u \rangle_{X} - F^{\ast}(p) + G(u) = \min_{u \in X}\, \max_{p \in Y}\,\, -\langle \nabla^{T}\,p, u \rangle_{X} - F^{\ast}(p) + G(u).
            \end{equation}
        Now, it remains to show how $F^{\ast}(p)$ looks like. From Example \ref{ex:legendre_fenchel_conjugate_example} 2. and the fact that the conjugate norm of the $l_{1}$ norm is the $l_{\infty}$ norm, we have
            \begin{equation}
                F^{\ast}(p) =
                    \begin{dcases*}
                        0 & \textnormal{if $||p||_{\infty} \le 1$,} \\
                        \infty & \textnormal{else},
                    \end{dcases*}
                \label{eq:rof_f_star}
            \end{equation}
        or equivalently $F^{\ast}(p) = \delta_{P}(p)$ for a set $P$ given by
            \begin{equation}
                P = \big\{ p \in Y : ||p||_{\infty} \le 1 \big\}.
                \label{eq:the_set_P}
            \end{equation}
        Using this notation in the saddle-point problem leads us to
            \begin{equation}
                \min_{u \in X}\, \max_{p \in Y}\,\, \langle p, \nabla\, u \rangle_{Y} + \frac{\lambda}{2} ||u - g||_{2}^{2} - \delta_{P}(p).
            \label{eq:primal_dual_rof_problem}
            \end{equation}
        We also want to propose the dual formluation of equation \ref{eq:primal_dual_rof_problem}. For that, we need to compute $G^{\ast}$. \\
        Setting $v = u - g$, which is equivalent to $u = v + g$. Further,
            \begin{eqnarray}
                G^{\ast}(p) &=& \sup_{u \in X} \langle u, p \rangle - \frac{\lambda}{2} ||u - g||_{2}^{2} \notag \\
                &=& \sup_{v \in X} \langle v + g, p \rangle - \frac{\lambda}{2} ||v||_{2}^{2} \notag \\
                &=& \underbrace{\sup_{v \in X} \langle v, p \rangle - \frac{\lambda}{2} ||v||_{2}^{2}}_{= \frac{1}{2\lambda} ||p||_{2}^{2} (\ast)} + \langle g, p \rangle \notag \\
                &=& \frac{1}{2\lambda} ||p||_{2}^{2} + \langle g, p \rangle. \notag
            \end{eqnarray}
        In $(\ast)$ we used example \ref{ex:legendre_fenchel_conjugate_example} 2. and 4.
        % By Example \ref{ex:legendre_fenchel_conjugate_example} 3. the conjugate of the function $G$ is given by $G^{\ast}(p) = \frac{\lambda}{2}||p - g||_{2}^{2}$ since the conjugate of the euclidean norm is itself the euclidean norm.
        Plugging $F^{\ast}$ and $G^{\ast}$, respectively, into equation \ref{eq:dual_problem} and setting $K^{\ast} = -\nabla^{T}$ we get
            \begin{equation}
                \max_{p \in Y}\,\, -(G^{\ast}(-K^{\ast}p) + F^{\ast}(p)) = \max_{p \in Y} - \bigg( \frac{1}{2\lambda} ||\nabla^{T}p||^{2}_{2} + \langle g, \nabla^{T}p \rangle_{X} + \delta_{P}(p) \bigg).
                % &=& \max_{p \in Y} -\bigg( \frac{\lambda}{2}||\nabla^{T}p + g||_{2}^{2} + \delta_{P}(p) \bigg)
            \label{eq:dual_rof_problem}
            \end{equation}
        which is the dual of the ROF model.
        %In \cite{Chambolle10afirst-order} one finds another notation for the dual ROF model, namely
        %     $$
        %         \max_{p \in Y} - \bigg( \frac{1}{2\lambda} ||\nabla^{T}p||^{2}_{2} + \langle g, \nabla^{T}p \rangle_{X} + \delta_{P}(p) \bigg).
        %     $$
        % This is equivalent to our formulation. First note that the parameter $\lambda$ in Equation \ref{eq:primal_rof_problem} can be swapped from the data fidelity term to the regularizer with $\frac{1}{\lambda}$. This has no effect on the energy. Computing $||\nabla^{T}p + g||_{2}^{2} = ||\nabla^{T}p||_{2}^{2} + ||g||_{2}^{2} + 2 \langle \nabla^{T}p, g \rangle$, gives us the first two terms of the above notation. The constant factor two vanishes by multiplying the objective function with $\frac{1}{2}$. And the term $||g||_{2}^{2}$ as a constant factor can be neglected, since it only shifts the energy. Then we derive the same notations as found in \cite{Chambolle10afirst-order}.

        % subsection rof_model_as_saddle_point_problem (end)

    \subsection{The Proximity Operators of the ROF Model} % (fold)
    \label{sub:the_proximity_operators_for_the_rof_model}

        In Algorithm \ref{alg:fast_primal_dual_algorithm} we saw, how saddle-point problems can be solved efficiently. In two of the three steps in this scheme we needed to calculate the proximity operators $(\textnormal{Id} + \sigma\,\partial\,F^{\ast})^{-1}$ and $(\textnormal{Id} + \tau\,\partial\,G)^{-1})$. Within the ROF Model one has $G(u) = \frac{\lambda}{2} ||u - g||_{2}^{2}$ and $F^{\ast}(p) = \delta_{P}(p)$. Applying Equation \ref{eq:proximity_operator} to $F^{\ast}$ we get
            $$
                (\textnormal{Id} + \sigma\,\partial\,F^{\ast})^{-1}(\tilde{p}) = \arg \min_{p \in P} \frac{||p - \tilde{p}||_{2}^{2}}{2} + \sigma\,\delta_{P}(p) = \arg \min_{p \in P} \frac{||p - \tilde{p}||_{2}^{2}}{2}.
            $$
        This is nothing but the euclidean projection of a vector $\tilde{p} \notin P$ onto the convex set $P$. From Example \ref{ex:projection_operator} 1. we have
            \begin{equation}
                p = (\textnormal{Id} + \sigma\,\partial\,F^{\ast})^{-1}(\tilde{p}) = \Pi_{P}(\tilde{p}) \Longleftrightarrow p_{i,j} = \frac{\tilde{p}_{i, j}}{\max(1, |\tilde{p}_{i, j}|)},
            \label{eq:proximity_operator_f_star_rof}
            \end{equation}
        pointwise for all $i = 1, ..., N, j = 1, ..., M$. So, it only remains to compute the proximity operator for our function $G$. Here, we get
            $$
                (\textnormal{Id} + \tau\,\partial\,G)^{-1}(\tilde{u}) = \arg \min_{u \in X} \frac{||u - \tilde{u}||_{2}^{2}}{2} + \frac{\tau\lambda}{2} ||u - g||_{2}^{2} = \arg \min_{u \in X} \mathcal{L}(u).
            $$
        Since this minimization problem is convex, a local minimizer is also a global minimizer of this problem. So, let $\hat{u} \in \arg \min\limits_{u \in X} \mathcal{L}(u)$. With proposition \ref{prop:zero_element_of_subgradient} we have
        % where $\mathcal{L}(u) = \frac{||u - \tilde{u}||_{2}^{2}}{2} + \frac{\tau\lambda}{2} ||u - g||_{2}^{2}$. Then this minimization problem is equivalent to compute $\nabla\mathcal{L}(u) = 0$. It follows
            $$
                0 \in \partial \mathcal{L}(\hat{u}) \Longleftrightarrow 0 \in (\hat{u} - \tilde{u}) + \tau\lambda (\hat{u} - g) \Longleftrightarrow (1 + \tau\lambda)\hat{u} = \tilde{u} + \tau\lambda g,
            $$
        and this implies $\hat{u} = \frac{\tilde{u} + \tau\lambda g}{1 + \tau\lambda}$.


        Overall, the following equivalence holds
        % We have for all $i = 1, ..., N$ and $j = 1, ..., M$ pointwise
            \begin{equation}
                u = (\textnormal{Id} + \tau\,\partial\,G)^{-1}(\tilde{u}) \Longleftrightarrow u_{i,j} = \frac{\tilde{u}_{i,j} + \tau\lambda g}{1 + \tau\sigma}
            \label{eq:proximity_operator_g_rof}
            \end{equation}

        for all $i = 1, ..., N$ and $j = 1, ..., M$ pointwise.

        As mentioned in section \ref{sec:discrete_setting}, each model has its proximity operators. Now, that we computed the versions for the ROF model, everything is set up and one can implement the primal-dual algorithm. Since the proximity operators of the ROF model have a very simple representation, all computations in the algorithm are highly parallelizable. We will discuss the implementation issues in chapter \ref{cha:applications_to_imaging}.
        % subsection the_proximity_operators_for_the_rof_model (end)

    % Now, that everything is defined and set perfectly one can implement the ROF model. This algorithm is massively parallelizable. We used in our framework C++ and CUDA respectively. We will discuss implementation issues and applications of the ROF Model in detail in Chapter \ref{cha:applications_to_imaging}.

% section the_rof_model (end)